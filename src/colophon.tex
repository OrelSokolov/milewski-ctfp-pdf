\lettrine[lraise=-0.03,loversize=0.08]{Э}{та книга} была скомпилирована \urlref{https://hmemcpy.com}{Игалем Табачником} путём преобразования оригинального текста Бартоша Милевски в формат \LaTeX{},
сначала извлекая оригинальные посты блога WordPress с помощью \urlref{https://mercury.postlight.com/web-parser/}{Mercury Web Parser}
для получения чистого HTML-содержимого, модифицируя и настраивая с помощью \urlref{https://www.crummy.com/software/BeautifulSoup/}{Beautiful Soup},
наконец, преобразуя в \LaTeX{} с помощью \urlref{https://pandoc.org/}{Pandoc}.

Шрифты: Linux Libertine для основного текста и Linux Biolinum для заголовков, оба от Филиппа Х. Полла. Моноширинный шрифт — Inconsolata,
созданный Рафом Левином и дополненный Димосфенисом Капонисом и Такаши Танигава в форме Inconsolata \acronym{LGC}. Шрифт обложки —
Alegreya, разработанный Хуаном Пабло дель Пералем.

Оригинальный дизайн макета книги и типографика выполнены Андресом Раба. Подсветка синтаксиса использует стиль ``GitHub'' для Pygments от
\urlref{https://github.com/hugomaiavieira/pygments-style-github}{Уго Майа Виеры}.
\ifdefined\OPTCustomLanguage{%
    \lettrine[lraise=-0.03,loversize=0.08]{Э}{та книга} была скомпилирована \urlref{https://hmemcpy.com}{Игалем Табачником} путём преобразования оригинального текста Бартоша Милевски в формат \LaTeX{},
сначала извлекая оригинальные посты блога WordPress с помощью \urlref{https://mercury.postlight.com/web-parser/}{Mercury Web Parser}
для получения чистого HTML-содержимого, модифицируя и настраивая с помощью \urlref{https://www.crummy.com/software/BeautifulSoup/}{Beautiful Soup},
наконец, преобразуя в \LaTeX{} с помощью \urlref{https://pandoc.org/}{Pandoc}.

Шрифты: Linux Libertine для основного текста и Linux Biolinum для заголовков, оба от Филиппа Х. Полла. Моноширинный шрифт — Inconsolata,
созданный Рафом Левином и дополненный Димосфенисом Капонисом и Такаши Танигава в форме Inconsolata \acronym{LGC}. Шрифт обложки —
Alegreya, разработанный Хуаном Пабло дель Пералем.

Оригинальный дизайн макета книги и типографика выполнены Андресом Раба. Подсветка синтаксиса использует стиль ``GitHub'' для Pygments от
\urlref{https://github.com/hugomaiavieira/pygments-style-github}{Уго Майа Виеры}.
\ifdefined\OPTCustomLanguage{%
    \lettrine[lraise=-0.03,loversize=0.08]{Э}{та книга} была скомпилирована \urlref{https://hmemcpy.com}{Игалем Табачником} путём преобразования оригинального текста Бартоша Милевски в формат \LaTeX{},
сначала извлекая оригинальные посты блога WordPress с помощью \urlref{https://mercury.postlight.com/web-parser/}{Mercury Web Parser}
для получения чистого HTML-содержимого, модифицируя и настраивая с помощью \urlref{https://www.crummy.com/software/BeautifulSoup/}{Beautiful Soup},
наконец, преобразуя в \LaTeX{} с помощью \urlref{https://pandoc.org/}{Pandoc}.

Шрифты: Linux Libertine для основного текста и Linux Biolinum для заголовков, оба от Филиппа Х. Полла. Моноширинный шрифт — Inconsolata,
созданный Рафом Левином и дополненный Димосфенисом Капонисом и Такаши Танигава в форме Inconsolata \acronym{LGC}. Шрифт обложки —
Alegreya, разработанный Хуаном Пабло дель Пералем.

Оригинальный дизайн макета книги и типографика выполнены Андресом Раба. Подсветка синтаксиса использует стиль ``GitHub'' для Pygments от
\urlref{https://github.com/hugomaiavieira/pygments-style-github}{Уго Майа Виеры}.
\ifdefined\OPTCustomLanguage{%
    \lettrine[lraise=-0.03,loversize=0.08]{Э}{та книга} была скомпилирована \urlref{https://hmemcpy.com}{Игалем Табачником} путём преобразования оригинального текста Бартоша Милевски в формат \LaTeX{},
сначала извлекая оригинальные посты блога WordPress с помощью \urlref{https://mercury.postlight.com/web-parser/}{Mercury Web Parser}
для получения чистого HTML-содержимого, модифицируя и настраивая с помощью \urlref{https://www.crummy.com/software/BeautifulSoup/}{Beautiful Soup},
наконец, преобразуя в \LaTeX{} с помощью \urlref{https://pandoc.org/}{Pandoc}.

Шрифты: Linux Libertine для основного текста и Linux Biolinum для заголовков, оба от Филиппа Х. Полла. Моноширинный шрифт — Inconsolata,
созданный Рафом Левином и дополненный Димосфенисом Капонисом и Такаши Танигава в форме Inconsolata \acronym{LGC}. Шрифт обложки —
Alegreya, разработанный Хуаном Пабло дель Пералем.

Оригинальный дизайн макета книги и типографика выполнены Андресом Раба. Подсветка синтаксиса использует стиль ``GitHub'' для Pygments от
\urlref{https://github.com/hugomaiavieira/pygments-style-github}{Уго Майа Виеры}.
\ifdefined\OPTCustomLanguage{%
    \input{content/\OPTCustomLanguage/colophon}
  }
\fi
  }
\fi
  }
\fi
  }
\fi