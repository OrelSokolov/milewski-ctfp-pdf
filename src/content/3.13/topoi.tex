% !TEX root = ../../ctfp-print.tex

\lettrine[lhang=0.17]{Я}{ понимаю, что мы можем} отдаляться от программирования и погружаться в
хардкорную математику. Но никогда не знаешь, что принесёт следующая большая революция в
программировании и какая математика может понадобиться для её
понимания. Есть несколько очень интересных идей вокруг, таких как
функциональное реактивное программирование с его непрерывным временем, расширение
системы типов Haskell зависимыми типами, или исследование
гомотопической теории типов в программировании.

До сих пор я небрежно отождествлял типы с \emph{множествами} значений.
Это не совсем правильно, потому что такой подход не принимает во внимание
тот факт, что в программировании мы \emph{вычисляем} значения, и
вычисление — это процесс, который занимает время и, в крайних случаях, может
не завершиться. Расходящиеся вычисления — часть каждого тьюринг-полного
языка.

Есть также фундаментальные причины, по которым теория множеств может быть не лучшим
выбором в качестве основы для компьютерной науки или даже самой математики. Хорошая
аналогия — теория множеств как язык ассемблера, который привязан
к конкретной архитектуре. Если вы хотите запустить свою математику на разных
архитектурах, нужно использовать более общие инструменты.

Одна возможность — использовать пространства вместо множеств. Пространства имеют больше
структуры и могут быть определены без обращения к множествам. Одна вещь,
обычно ассоциируемая с пространствами — топология, которая необходима для определения
таких вещей, как непрерывность. И традиционный подход к топологии,
вы угадали, через теорию множеств. В частности, понятие
подмножества центрально для топологии. Неудивительно, что теоретики категорий
обобщили эту идею на категории, отличные от $\Set$. Тип
категории, который имеет как раз правильные свойства, чтобы служить заменой
теории множеств, называется \newterm{топосом} (множественное число: топосы), и он
предоставляет, среди прочего, обобщённое понятие подмножества.

\section{Подобъектный классификатор}

Начнём с попытки выразить идею подмножества, используя функции,
а не элементы. Любая функция $f$ из некоторого множества $a$
в $b$ определяет подмножество $b$ — то, которое является образом
$a$ при $f$. Но есть много функций, которые определяют
одно и то же подмножество. Нам нужно быть более конкретными. Для начала можно
сосредоточиться на функциях, которые инъективны --- тех, которые не сжимают множество
элементов в один. Инъективные функции ``инъецируют'' одно множество в другое.
Для конечных множеств можно визуализировать инъективные функции как параллельные
стрелки, соединяющие элементы одного множества с элементами другого. Конечно,
первое множество не может быть больше второго множества, иначе стрелки обязательно
сойдутся. Всё ещё остаётся некоторая неоднозначность: может быть
другое множество $a'$ и другая инъективная функция
$f'$ из этого множества в $b$, которая выбирает то же самое
подмножество. Но вы легко убедитесь, что такое множество должно было бы
быть изоморфным $a$. Мы можем использовать этот факт для определения подмножества
как семейства инъективных функций, связанных изоморфизмами их
областей определения. Точнее, мы говорим, что две инъективные функции:
\begin{align*}
  f  & \Colon a \to b  \\
  f' & \Colon a' \to b
\end{align*}
эквивалентны, если существует изоморфизм:
\[h \Colon a \to a'\]
такой, что:
\[f = f' \circ h\]
Такое семейство эквивалентных инъекций определяет подмножество $b$.

\begin{figure}[H]
  \centering
  \includegraphics[width=0.4\textwidth]{images/subsetinjection.jpg}
\end{figure}

\noindent
Это определение можно поднять на произвольную категорию, если заменить
инъективные функции мономорфизмами. Просто напомню, что
мономорфизм $m$ из $a$ в $b$ определяется своим
универсальным свойством. Для любого объекта $c$ и любой пары морфизмов:
\begin{align*}
  g  & \Colon c \to a \\
  g' & \Colon c \to a
\end{align*}
таких, что:
\[m \circ g = m \circ g'\]
должно быть, что $g = g'$.

\begin{figure}[H]
  \centering
  \includegraphics[width=0.4\textwidth]{images/monomorphism.jpg}
\end{figure}

\noindent
На множествах это определение легче понять, если рассмотрим, что бы
означало для функции $m$ \emph{не} быть мономорфизмом. Она бы
отображала два различных элемента $a$ в один элемент
$b$. Тогда могли бы найти две функции $g$ и
$g'$, которые отличаются только в этих двух элементах.
Посткомпозиция с $m$ тогда скрыла бы это различие.

\begin{figure}[H]
  \centering
  \includegraphics[width=0.4\textwidth]{images/notmono.jpg}
\end{figure}

\noindent
Есть другой способ определения подмножества: использовать одну функцию,
называемую характеристической функцией. Это функция $\chi$ из
множества $b$ в двухэлементное множество $\Omega$. Один элемент этого множества
обозначается как ``истина'', а другой как ``ложь''. Эта функция
присваивает ``истину'' тем элементам $b$, которые являются членами
подмножества, и ``ложь'' тем, которые нет.

Остаётся указать, что означает обозначение элемента
$\Omega$ как ``истина''. Можем использовать стандартный трюк: использовать функцию
из синглтонного множества в $\Omega$. Назовём эту функцию
$\mathit{true}$:
\[\mathit{true} \Colon 1 \to \Omega\]

\begin{figure}[H]
  \centering
  \includegraphics[width=0.4\textwidth]{images/true.jpg}
\end{figure}

\noindent
Эти определения можно объединить таким образом, что они не только
определяют, что такое подобъект, но также определяют специальный объект
$\Omega$, не говоря об элементах. Идея в том, что мы хотим, чтобы
морфизм $\mathit{true}$ представлял ``типовой'' подобъект. В
$\Set$ он выбирает одноэлементное подмножество из двухэлементного множества
$\Omega$. Это так же типично, как только возможно. Это явно собственное подмножество,
потому что $\Omega$ имеет ещё один элемент, который \emph{не} в этом
подмножестве.

В более общей обстановке мы определяем $\mathit{true}$ как мономорфизм
из терминального объекта в \emph{классифицирующий объект} $\Omega$.
Но мы должны определить классифицирующий объект. Нам нужно универсальное
свойство, которое связывает этот объект с характеристической функцией. Оказывается,
что в $\Set$ откат $\mathit{true}$ вдоль
характеристической функции $\chi$ определяет как подмножество $a$,
так и инъективную функцию, которая вкладывает его в $b$. Вот
диаграмма отката:

\begin{figure}[H]
  \centering
  \includegraphics[width=0.4\textwidth]{images/pullback.jpg}
\end{figure}

\noindent
Давайте проанализируем эту диаграмму. Уравнение отката:
\[\mathit{true} \circ \mathit{unit} = \chi \circ f\]
Функция $\mathit{true} \circ \mathit{unit}$ отображает каждый элемент $a$ в
``истину''. Следовательно, $f$ должна отображать все элементы $a$ в
те элементы $b$, для которых $\chi$ даёт ``истину''. Это,
по определению, элементы подмножества, которое задаётся
характеристической функцией $\chi$. Таким образом, образ $f$ действительно является
рассматриваемым подмножеством. Универсальность отката гарантирует, что
$f$ инъективна.

Эта диаграмма отката может использоваться для определения классифицирующего объекта в
категориях, отличных от $\Set$. Такая категория должна иметь терминальный
объект, что позволит нам определить мономорфизм $\mathit{true}$. Она также должна
иметь откаты --- фактическое требование состоит в том, что она должна иметь все
конечные пределы (откат является примером конечного предела). При этих
предположениях мы определяем классифицирующий объект $\Omega$ свойством,
что для каждого мономорфизма $f$ существует уникальный морфизм
$\chi$, который завершает диаграмму отката.

Давайте проанализируем последнее утверждение. Когда мы строим откат, нам
даны три объекта $\Omega$, $b$ и $1$; и два
морфизма, $\mathit{true}$ и $\chi$. Существование отката
означает, что можем найти наилучший такой объект $a$, оснащённый
двумя морфизмами $f$ и $\mathit{unit}$ (последний однозначно
определяется определением терминального объекта), которые делают
диаграмму коммутативной.

Здесь мы решаем другую систему уравнений. Мы решаем для
$\Omega$ и $\mathit{true}$, варьируя как $a$, \emph{так и}
$b$. Для данных $a$ и $b$ может быть или не
быть мономорфизма $f \Colon a \to b$. Но если он есть, мы
хотим, чтобы он был откатом некоторого $\chi$. Более того, мы хотим, чтобы этот
$\chi$ однозначно определялся $f$.

Мы не можем сказать, что существует взаимно однозначное соответствие между
мономорфизмами $f$ и характеристическими функциями $\chi$,
потому что откат уникален только с точностью до изоморфизма. Но помните наше
более раннее определение подмножества как семейства эквивалентных инъекций. Мы
можем обобщить его, определив подобъект $b$ как семейство
эквивалентных мономорфизмов в $b$. Это семейство мономорфизмов находится
во взаимно однозначном соответствии с семейством эквивалентных откатов
нашей диаграммы.

Таким образом, мы можем определить множество подобъектов $b$, $\mathit{Sub}(b)$,
как семейство мономорфизмов, и увидеть, что оно изоморфно множеству
морфизмов из $b$ в $\Omega$:
\[\mathit{Sub}(b) \cong \cat{C}(b, \Omega)\]
Оказывается, это естественный изоморфизм двух функторов. Другими
словами, $\mathit{Sub}(-)$ — это представимый (контравариантный) функтор, чьим
представлением является объект $\Omega$.

\section{Топос}

Топос — это категория, которая:

\begin{enumerate}
  \tightlist
  \item
        Декартово замкнута: Имеет все произведения, терминальный объект и
        экспоненты (определённые как правые сопряжённые к произведениям),
  \item
        Имеет пределы для всех конечных диаграмм,
  \item
        Имеет подобъектный классификатор $\Omega$.
\end{enumerate}

Этот набор свойств делает топос претендентом на замену $\Set$ в большинстве
приложений. Он также имеет дополнительные свойства, которые следуют из его
определения. Например, топос имеет все конечные копределы, включая
начальный объект.

Было бы заманчиво определить подобъектный классификатор как копроизведение
(сумму) двух копий терминального объекта --- это то, что он есть в
$\Set$ --- но мы хотим быть более общими. Топосы, в которых
это верно, называются булевыми.

\section{Топосы и логика}

В теории множеств характеристическая функция может интерпретироваться как определяющая
свойство элементов множества --- \newterm{предикат}, который истинен
для некоторых элементов и ложен для других. Предикат $\mathit{isEven}$
выбирает подмножество чётных чисел из множества натуральных чисел. В
топосе можем обобщить идею предиката как морфизма из
объекта $a$ в $\Omega$. Вот почему $\Omega$ иногда
называют объектом истины.

Предикаты — строительные блоки логики. Топос содержит весь
необходимый инструментарий для изучения логики. Он имеет произведения, которые
соответствуют логическим конъюнкциям (логическое \emph{и}), копроизведения для
дизъюнкций (логическое \emph{или}) и экспоненты для импликаций. Все
стандартные аксиомы логики выполняются в топосе, за исключением закона исключённого
третьего (или, эквивалентно, устранения двойного отрицания). Поэтому
логика топоса соответствует конструктивной или интуиционистской логике.

Интуиционистская логика неуклонно набирает силу, находя
неожиданную поддержку со стороны компьютерной науки. Классическое понятие
исключённого третьего основано на убеждении, что существует абсолютная истина: Любое
утверждение либо истинно, либо ложно, или, как говорили древние римляне,
\emph{tertium non datur} (третьего не дано). Но единственный способ, которым мы
можем узнать, истинно что-то или ложно, — если сможем доказать или
опровергнуть это. Доказательство — это процесс, вычисление --- и мы знаем, что
вычисления занимают время и ресурсы. В некоторых случаях они могут никогда
не завершиться. Не имеет смысла утверждать, что утверждение истинно, если мы
не можем доказать его за конечное время. Топос с его более нюансированным
объектом истины предоставляет более общую структуру для моделирования интересных
логик.

\section{Задачи}

\begin{enumerate}
  \tightlist
  \item
        Покажите, что функция $f$, которая является откатом
        $\mathit{true}$ вдоль характеристической функции, должна быть инъективной.
\end{enumerate}
