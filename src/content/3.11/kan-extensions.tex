% !TEX root = ../../ctfp-print.tex

\lettrine[lhang=0.17]{Д}{о сих пор мы} в основном работали с одной категорией или парой
категорий. В некоторых случаях это было слишком ограничивающим.

Например, при определении предела в категории $\cat{C}$ мы вводили
индексную категорию $\cat{I}$ как шаблон для паттерна, который
формировал бы основу для наших конусов. Имело бы смысл ввести
другую категорию, тривиальную, чтобы служить шаблоном для вершины
конуса. Вместо этого мы использовали константный функтор $\Delta_c$ из
$\cat{I}$ в $\cat{C}$.

Пришло время исправить эту неуклюжесть. Давайте определим предел, используя три
категории. Начнём с функтора $D$ из индексной
категории $\cat{I}$ в $\cat{C}$. Это функтор, который выбирает основание
конуса --- функтор диаграммы.

\begin{figure}[H]
  \centering
  \includegraphics[width=0.4\textwidth]{images/kan2.jpg}
\end{figure}

\noindent
Новое добавление — это категория $\cat{1}$, которая содержит один
объект (и один тождественный морфизм). Существует только один возможный
функтор $K$ из $\cat{I}$ в эту категорию. Он отображает все объекты
в единственный объект в $\cat{1}$, и все морфизмы в тождественный
морфизм. Любой функтор $F$ из $\cat{1}$ в $\cat{C}$ выбирает
потенциальную вершину для нашего конуса.

\begin{figure}[H]
  \centering
  \includegraphics[width=0.4\textwidth]{images/kan15.jpg}
\end{figure}

\noindent
Конус — это естественное преобразование $\varepsilon$ из $F \circ K$ в
$D$. Заметим, что $F \circ K$ делает в точности то же самое, что и
наш исходный $\Delta_c$. Следующая диаграмма показывает это
преобразование.

\begin{figure}[H]
  \centering
  \includegraphics[width=0.4\textwidth]{images/kan3-e1492120491591.jpg}
\end{figure}

\noindent
Теперь мы можем определить универсальное свойство, которое выбирает ``лучший'' такой
функтор $F$. Этот $F$ отобразит $\cat{1}$ в объект,
который является пределом $D$ в $\cat{C}$, и естественное
преобразование $\varepsilon$ из $F \circ K$ в $D$ даст
соответствующие проекции. Этот универсальный функтор называется
правым расширением Кана $D$ вдоль $K$ и обозначается
$\Ran_{K}D$.

Давайте сформулируем универсальное свойство. Предположим, у нас есть другой конус ---
то есть другой функтор $F'$ вместе с естественным
преобразованием $\varepsilon'$ из $F' \circ K$ в
$D$.

\begin{figure}[H]
  \centering
  \includegraphics[width=0.4\textwidth]{images/kan31-e1492120512209.jpg}
\end{figure}

\noindent
Если расширение Кана $F = Ran_{K}D$ существует, должно существовать единственное
естественное преобразование $\sigma$ из $F'$ в него, такое,
что $\varepsilon'$ факторизуется через $\varepsilon$, то есть:
\[\varepsilon' = \varepsilon \cdot (\sigma \circ K)\]
Здесь $\sigma \circ K$ — это горизонтальная композиция двух естественных
преобразований (одно из них является тождественным естественным преобразованием
на $K$). Это преобразование затем вертикально композируется с
$\varepsilon$.

\begin{figure}[H]
  \centering
  \includegraphics[width=0.4\textwidth]{images/kan5.jpg}
\end{figure}

\noindent
По компонентам, при действии на объект $i$ в $\cat{I}$, мы получаем:
\[\varepsilon'_i = \varepsilon_i \circ \sigma_{K i}\]
В нашем случае $\sigma$ имеет только одну компоненту, соответствующую
единственному объекту $\cat{1}$. Таким образом, это действительно единственный морфизм
из вершины конуса, определённого $F'$, в вершину
универсального конуса, определённого $\Ran_{K}D$. Коммутативные условия
в точности совпадают с теми, которые требуются определением предела.

Но, что важно, мы вольны заменить тривиальную категорию $\cat{1}$
произвольной категорией $\cat{A}$, и определение правого расширения
Кана останется верным.

\section{Правое расширение Кана}

Правое расширение Кана функтора $D \Colon \cat{I} \to \cat{C}$
вдоль функтора $K \Colon \cat{I} \to \cat{A}$ — это функтор
$F \Colon \cat{A} \to \cat{C}$ (обозначаемый $\Ran_{K}D$) вместе с
естественным преобразованием
\[\varepsilon \Colon F \circ K \to D\]
таким, что для любого другого функтора $F' \Colon \cat{A} \to \cat{C}$ и
естественного преобразования
\[\varepsilon' \Colon F' \circ K \to D\]
существует единственное естественное преобразование
\[\sigma \Colon F' \to F\]
которое факторизует $\varepsilon'$:
\[\varepsilon' = \varepsilon \cdot (\sigma \circ K)\]
Это довольно запутанно, но можно визуализировать на этой хорошей диаграмме:

\begin{figure}[H]
  \centering
  \includegraphics[width=0.4\textwidth]{images/kan7.jpg}
\end{figure}

\noindent
Интересный способ взглянуть на это — заметить, что в некотором смысле
расширение Кана действует как обратное к ``умножению функтора''. Некоторые
авторы заходят так далеко, что используют обозначение $D/K$ для $\Ran_{K}D$.
Действительно, в этих обозначениях определение $\varepsilon$, которое также
называется коединицей правого расширения Кана, выглядит как простое
сокращение:
\[\varepsilon \Colon D/K \circ K \to D\]
Существует другая интерпретация расширений Кана. Рассмотрим, что
функтор $K$ вкладывает категорию $\cat{I}$ внутрь $\cat{A}$. В
простейшем случае $\cat{I}$ может быть просто подкатегорией $\cat{A}$. У нас есть
функтор $D$, который отображает $\cat{I}$ в $\cat{C}$. Можем ли мы расширить
$D$ до функтора $F$, который определён на всём
$\cat{A}$? В идеале такое расширение сделало бы композицию
$F \circ K$ изоморфной $D$. Другими словами, $F$
расширял бы область определения $D$ до $\cat{A}$. Но
полноценный изоморфизм обычно слишком многого требует, и мы можем обойтись
только половиной его, а именно односторонним естественным преобразованием $\varepsilon$ из
$F \circ K$ в $D$. (Левое расширение Кана выбирает другое направление.)

\begin{figure}[H]
  \centering
  \includegraphics[width=0.4\textwidth]{images/kan6.jpg}
\end{figure}

\noindent
Конечно, картина вложения рушится, когда функтор $K$
не инъективен на объектах или не точен на hom-множествах, как в
примере предела. В этом случае расширение Кана делает всё возможное, чтобы
экстраполировать потерянную информацию.

\section{Расширение Кана как сопряжение}

Теперь предположим, что правое расширение Кана существует для любого $D$ (и
фиксированного $K$). В этом случае $\Ran_{K}-$ (с тире,
заменяющим $D$) является функтором из категории функторов
${[}\cat{I}, \cat{C}{]}$ в категорию функторов ${[}\cat{A}, \cat{C}{]}$. Оказывается,
что этот функтор является правым сопряжённым к функтору предкомпозиции
$- \circ K$. Последний отображает функторы из ${[}\cat{A}, \cat{C}{]}$
в функторы из ${[}\cat{I}, \cat{C}{]}$. Сопряжение:
\[[\cat{I}, \cat{C}](F' \circ K, D) \cong [\cat{A}, \cat{C}](F', \Ran_{K}D)\]
Это просто переформулировка факта, что каждому естественному
преобразованию, которое мы назвали $\varepsilon'$, соответствует единственное естественное
преобразование, которое мы назвали $\sigma$.

\begin{figure}[H]
  \centering
  \includegraphics[width=0.4\textwidth]{images/kan92.jpg}
\end{figure}

\noindent
Более того, если мы выберем категорию $\cat{I}$ такой же, как
$\cat{C}$, мы можем подставить тождественный функтор $I_{\cat{C}}$ вместо
$D$. Мы получим следующее тождество:
\[[\cat{C}, \cat{C}](F' \circ K, I_{\cat{C}}) \cong [\cat{A}, \cat{C}](F', \Ran_{K}I_{\cat{C}})\]
Теперь мы можем выбрать $F'$ таким же, как $\Ran_{K}I_{\cat{C}}$. В
этом случае правая часть содержит тождественное естественное
преобразование, и, соответственно, левая часть даёт нам
следующее естественное преобразование:
\[\varepsilon \Colon \Ran_{K}I_{\cat{C}} \circ K \to I_{\cat{C}}\]
Это очень похоже на коединицу сопряжения:
\[\Ran_{K}I_{\cat{C}} \dashv K\]
Действительно, правое расширение Кана тождественного функтора вдоль функтора
$K$ может использоваться для вычисления левого сопряжённого к $K$. Для
этого необходимо ещё одно условие: правое расширение Кана должно быть
сохранено функтором $K$. Сохранение расширения
означает, что если мы вычислим расширение Кана функтора, предкомпонированного
с $K$, мы должны получить тот же результат, что и при предкомпозиции
исходного расширения Кана с $K$. В нашем случае это условие
упрощается до:
\[K \circ \Ran_{K}I_{\cat{C}} \cong \Ran_{K}K\]
Заметим, что, используя обозначение деления на $K$, сопряжение можно
записать как:
\[I/K \dashv K\]
что подтверждает нашу интуицию, что сопряжение описывает некоторый вид
обратного. Условие сохранения становится:
\[K \circ I/K \cong K/K\]
Правое расширение Кана функтора вдоль самого себя, $K/K$,
называется моно́идой коплотности.

Формула сопряжения — важный результат, потому что, как мы скоро увидим,
мы можем вычислять расширения Кана, используя энды (коэнды), таким образом получая
практические средства нахождения правых (и левых) сопряжённых.

\section{Левое расширение Кана}

Существует двойственная конструкция, которая даёт нам левое расширение Кана. Чтобы
построить некоторую интуицию, мы начнём с определения копредела
и перестроим его, чтобы использовать синглтонную категорию $\cat{1}$. Мы строим
коконус, используя функтор $D \Colon \cat{I} \to \cat{C}$ для формирования его
основания, и функтор $F \Colon \cat{1} \to \cat{C}$ для выбора его вершины.

\begin{figure}[H]
  \centering
  \includegraphics[width=0.4\textwidth]{images/kan81.jpg}
\end{figure}

\noindent
Стороны коконуса, инъекции, являются компонентами естественного
преобразования $\eta$ из $D$ в $F \circ K$.

\begin{figure}[H]
  \centering
  \includegraphics[width=0.4\textwidth]{images/kan10a.jpg}
\end{figure}

\noindent
Копредел — это универсальный коконус. Поэтому для любого другого функтора
$F'$ и естественного преобразования
\[\eta' \Colon D \to F' \circ K\]

\begin{figure}[H]
  \centering
  \includegraphics[width=0.4\textwidth]{images/kan10b.jpg}
\end{figure}

\noindent
существует единственное естественное преобразование $\sigma$ из $F$ в $F'$

\begin{figure}[H]
  \centering
  \includegraphics[width=0.4\textwidth]{images/kan14.jpg}
\end{figure}

\noindent
такое, что:
\[\eta' = (\sigma \circ K) \cdot \eta\]
Это иллюстрируется следующей диаграммой:

\begin{figure}[H]
  \centering
  \includegraphics[width=0.4\textwidth]{images/kan112.jpg}
\end{figure}

\noindent
Заменяя синглтонную категорию $\cat{1}$ на $\cat{A}$, это
определение естественно обобщается до определения левого расширения Кана,
обозначаемого $\Lan_{K}D$.

\begin{figure}[H]
  \centering
  \includegraphics[width=0.4\textwidth]{images/kan12.jpg}
\end{figure}

\noindent
Естественное преобразование:
\[\eta \Colon D \to \Lan_{K}D \circ K\]
называется единицей левого расширения Кана.

Как и раньше, мы можем переформулировать взаимно однозначное соответствие между естественными
преобразованиями:
\[\eta' = (\sigma \circ K) \cdot \eta\]
в терминах сопряжения:
\[[\cat{A}, \cat{C}](\Lan_{K}D, F') \cong [\cat{I}, \cat{C}](D, F' \circ K)\]
Другими словами, левое расширение Кана — это левый сопряжённый функтор, а
правое расширение Кана — это правый сопряжённый функтор предкомпозиции с
$K$.

Точно так же, как правое расширение Кана тождественного функтора могло использоваться
для вычисления левого сопряжённого к $K$, левое расширение Кана
тождественного функтора оказывается правым сопряжённым к $K$
(где $\eta$ является единицей сопряжения):
\[K \dashv \Lan_{K}I_{\cat{C}}\]
Объединяя два результата, получаем:
\[\Ran_{K}I_{\cat{C}} \dashv K \dashv \Lan_{K}I_{\cat{C}}\]

\section{Расширения Кана как энды}

Настоящая мощь расширений Кана происходит из того факта, что они могут быть
вычислены с использованием энд (и коэнд). Для простоты мы ограничим наше
внимание случаем, когда целевая категория $\cat{C}$ — это
$\Set$, но формулы можно распространить на любую категорию.

Давайте вернёмся к идее, что расширение Кана может использоваться для расширения
действия функтора за пределы его исходной области определения. Предположим, что
$K$ вкладывает $\cat{I}$ в $\cat{A}$. Функтор $D$ отображает
$\cat{I}$ в $\Set$. Мы могли бы просто сказать, что для любого объекта
$a$ в образе $K$, то есть $a = K i$,
расширенный функтор отображает $a$ в $D i$. Проблема в том, что
делать с теми объектами в $\cat{A}$, которые находятся вне образа
$K$? Идея в том, что каждый такой объект потенциально связан
через множество морфизмов с каждым объектом в образе $K$. Функтор
должен сохранять эти морфизмы. Совокупность морфизмов из
объекта $a$ в образ $K$ характеризуется
hom-функтором:
\[\cat{A}(a, K -)\]

\begin{figure}[H]
  \centering
  \includegraphics[width=0.4\textwidth]{images/kan13.jpg}
\end{figure}

\noindent
Заметим, что этот hom-функтор — это композиция двух функторов:
\[\cat{A}(a, K -) = \cat{A}(a, -) \circ K\]
Правое расширение Кана — это правый сопряжённый функтор композиции:
\[[\cat{I}, \Set](F' \circ K, D) \cong [\cat{A}, \Set](F', \Ran_{K}D)\]
Посмотрим, что произойдёт, если заменить $F'$ на hom-функтор:
\[[\cat{I}, \Set](\cat{A}(a, -) \circ K, D) \cong [\cat{A}, \Set](\cat{A}(a, -), \Ran_{K}D)\]
а затем подставить композицию:
\[[\cat{I}, \Set](\cat{A}(a, K -), D) \cong [\cat{A}, \Set](\cat{A}(a, -), \Ran_{K}D)\]
Правую часть можно упростить с помощью леммы Йонеды:
\[[\cat{I}, \Set](\cat{A}(a, K -), D) \cong \Ran_{K}D a\]
Теперь мы можем переписать множество естественных преобразований как энду, чтобы получить
эту очень удобную формулу для правого расширения Кана:
\[\Ran_{K}D a \cong \int_i \Set(\cat{A}(a, K i), D i)\]
Существует аналогичная формула для левого расширения Кана в терминах
коэнды:
\[\Lan_{K}D a = \int^i \cat{A}(K i, a)\times{}D i\]
Чтобы убедиться, что это так, мы покажем, что это действительно левый
сопряжённый функтор к композиции:
\[[\cat{A}, \Set](\Lan_{K}D, F') \cong [\cat{I}, \Set](D, F' \circ K)\]
Подставим нашу формулу в левую часть:
\[[\cat{A}, \Set](\int^i \cat{A}(K i, -)\times{}D i, F')\]
Это множество естественных преобразований, поэтому его можно переписать как
энду:
\[\int_a \Set(\int^i \cat{A}(K i, a)\times{}D i, F' a)\]
Используя непрерывность hom-функтора, можем заменить коэнду
эндой:
\[\int_a \int_i \Set(\cat{A}(K i, a)\times{}D i, F' a)\]
Можем использовать сопряжение произведение-экспонента:
\[\int_a \int_i \Set(\cat{A}(K i, a),\ (F' a)^{D i})\]
Экспонента изоморфна соответствующему hom-множеству:
\[\int_a \int_i \Set(\cat{A}(K i, a),\ \Set(D i, F' a))\]
Существует теорема, называемая теоремой Фубини, которая позволяет поменять местами
две энды:
\[\int_i \int_a \Set(\cat{A}(K i, a),\ \Set(D i, F' a))\]
Внутренняя энда представляет множество естественных преобразований между двумя
функторами, поэтому можем использовать лемму Йонеды:
\[\int_i \Set(D i, F' (K i))\]
Это действительно множество естественных преобразований, которое образует правую
часть сопряжения, которое мы хотели доказать:
\[[\cat{I}, \Set](D, F' \circ K)\]
Такие вычисления с использованием энд, коэнд и леммы Йонеды
достаточно типичны для ``исчисления'' энд.

\section{Расширения Кана в Haskell}

Формулы энд/коэнд для расширений Кана легко переводятся на
Haskell. Начнём с правого расширения:
\[\Ran_{K}D a \cong \int_i \Set(\cat{A}(a, K i), D i)\]
Заменим энду универсальным квантором, а hom-множества —
типами функций:

\src{snippet01}
Глядя на это определение, ясно, что \code{Ran} должен содержать
значение типа \code{a}, к которому можно применить функцию, и
естественное преобразование между двумя функторами \code{k} и
\code{d}. Например, предположим, что \code{k} — это функтор дерева,
а \code{d} — функтор списка, и вам дан
\code{Ran Tree {[}{]} String}. Если передать ему функцию:

\src{snippet02}
вы получите обратно список \code{Int}, и так далее. Правое расширение
Кана использует вашу функцию для создания дерева, а затем упаковывает его
в список. Например, вы можете передать ему парсер, который генерирует
дерево разбора из строки, и получите список, соответствующий
обходу этого дерева в глубину.

Правое расширение Кана можно использовать для вычисления левого сопряжённого
данного функтора, заменив функтор \code{d} тождественным
функтором. Это приводит к тому, что левый сопряжённый функтора \code{k} представляется
множеством полиморфных функций типа:

\src{snippet03}
Предположим, что \code{k} — это забывающий функтор из категории
моноидов. Тогда универсальный квантор проходит по всем моноидам. Конечно,
в Haskell мы не можем выразить законы моноидов, но следующее является
приличным приближением результирующего свободного функтора (забывающий
функтор \code{k} тождественен на объектах):

\src{snippet04}
Как и ожидалось, это генерирует свободные моноиды, или списки Haskell:

\src{snippet05}
Левое расширение Кана — это коэнда:
\[\Lan_{K}D a = \int^i \cat{A}(K i, a)\times{}D i\]
поэтому оно переводится в экзистенциальный квантор. Символически:

\begin{snip}{haskell}
Lan k d a = exists i. (k i -> a, d i)
\end{snip}
Это можно закодировать в Haskell с помощью \acronym{GADT}ов или используя универсально
квантифицированный конструктор данных:

\src{snippet06}
Интерпретация этой структуры данных заключается в том, что она содержит функцию,
которая берёт контейнер каких-то неуказанных \code{i} и производит
\code{a}. Она также имеет контейнер этих \code{i}. Поскольку вы
не имеете представления, что такое \code{i}, единственное, что можно сделать с этой структурой данных, — это извлечь контейнер \code{i}, упаковать его в
контейнер, определённый функтором \code{k}, используя естественное
преобразование, и вызвать функцию для получения \code{a}. Например,
если \code{d} — дерево, а \code{k} — список, можно
сериализовать дерево, вызвать функцию с результирующим списком и
получить \code{a}.

Левое расширение Кана можно использовать для вычисления правого сопряжённого
функтора. Мы знаем, что правый сопряжённый функтора произведения — это
экспонента, поэтому попробуем реализовать её с помощью расширения Кана:

\src{snippet07}
Это действительно изоморфно типу функции, что подтверждается
следующей парой функций:

\src{snippet08}
Заметим, что, как описано ранее в общем случае, мы выполнили
следующие шаги:

\begin{enumerate}
  \tightlist
  \item
        Извлекли контейнер \code{x} (здесь это
        просто тривиальный тождественный контейнер) и функцию \code{f}.
  \item
        Переупаковали контейнер, используя естественное преобразование между
        тождественным функтором и функтором пары.
  \item
        Вызвали функцию \code{f}.
\end{enumerate}

\section{Свободный функтор}

Интересное приложение расширений Кана — построение
свободного функтора. Это решение следующей практической проблемы:
предположим, у вас есть конструктор типов --- то есть отображение объектов. Возможно ли
определить функтор на основе этого конструктора типов? Другими
словами, можем ли мы определить отображение морфизмов, которое расширит этот конструктор типов до полноценного эндофунктора?

Ключевое наблюдение состоит в том, что конструктор типов можно описать как
функтор, область определения которого — дискретная категория. Дискретная категория не имеет
морфизмов, кроме тождественных морфизмов. Для данной категории $\cat{C}$
мы всегда можем построить дискретную категорию $\cat{|C|}$,
просто отбросив все нетождественные морфизмы. Функтор $F$
из $\cat{|C|}$ в $\cat{C}$ тогда является простым отображением
объектов, или тем, что мы называем конструктором типов в Haskell. Также существует
канонический функтор $J$, который вкладывает $\cat{|C|}$
в $\cat{C}$: он тождественен на объектах (и на тождественных морфизмах).
Левое расширение Кана $F$ вдоль $J$, если оно существует,
тогда является функтором из $\cat{C}$ в $\cat{C}$:
\[\Lan_{J}F a = \int^i \cat{C}(J i, a)\times{}F i\]
Это называется свободным функтором, основанным на $F$.

В Haskell мы бы написали это так:

\src{snippet09}
Действительно, для любого конструктора типов \code{f}, \code{FreeF f} является
функтором:

\src{snippet10}
Как видите, свободный функтор имитирует поднятие функции,
записывая и саму функцию, и её аргумент. Он накапливает поднятые
функции, записывая их композицию. Законы функтора
автоматически выполняются. Это построение использовалось в статье
\urlref{http://okmij.org/ftp/Haskell/extensible/more.pdf}{Freer Monads,
  More Extensible Effects}.

Альтернативно, мы можем использовать правое расширение Кана для той же цели:

\src{snippet11}
Легко проверить, что это действительно функтор:

\src{snippet12}
