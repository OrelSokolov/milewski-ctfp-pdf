% !TEX root = ../../ctfp-print.tex

\lettrine[lhang=0.17]{S}{o far we've been} mostly working with a single category or a pair of
categories. In some cases that was a little too constraining.

For instance, when defining a limit in a category $\cat{C}$, we introduced an
index category $\cat{I}$ as the template for the pattern that would
form the basis for our cones. It would have made sense to introduce
another category, a trivial one, to serve as a template for the apex of
the cone. Instead we used the constant functor $\Delta_c$ from
$\cat{I}$ to $\cat{C}$.

It's time to fix this awkwardness. Let's define a limit using three
categories. Let's start with the functor $D$ from the index
category $\cat{I}$ to $\cat{C}$. This is the functor that selects the base
of the cone --- the diagram functor.

\begin{figure}[H]
  \centering
  \includegraphics[width=0.4\textwidth]{images/kan2.jpg}
\end{figure}

\noindent
The new addition is the category $\cat{1}$ that contains a single
object (and a single identity morphism). There is only one possible
functor $K$ from $\cat{I}$ to this category. It maps all objects
to the only object in $\cat{1}$, and all morphisms to the identity
morphism. Any functor $F$ from $\cat{1}$ to $\cat{C}$ picks a
potential apex for our cone.

\begin{figure}[H]
  \centering
  \includegraphics[width=0.4\textwidth]{images/kan15.jpg}
\end{figure}

\noindent
A cone is a natural transformation $\varepsilon$ from $F \circ K$ to
$D$. Notice that $F \circ K$ does exactly the same thing as
our original $\Delta_c$. The following diagram shows this
transformation.

\begin{figure}[H]
  \centering
  \includegraphics[width=0.4\textwidth]{images/kan3-e1492120491591.jpg}
\end{figure}

\noindent
We can now define a universal property that picks the ``best'' such
functor $F$. This $F$ will map $\cat{1}$ to the object
that is the limit of $D$ in $\cat{C}$, and the natural
transformation $\varepsilon$ from $F \circ K$ to $D$ will
provide the corresponding projections. This universal functor is called
the right Kan extension of $D$ along $K$ and is denoted by
$\Ran_{K}D$.

Let's formulate the universal property. Suppose we have another cone ---
that is another functor $F'$ together with a natural
transformation $\varepsilon'$ from $F' \circ K$ to
$D$.

\begin{figure}[H]
  \centering
  \includegraphics[width=0.4\textwidth]{images/kan31-e1492120512209.jpg}
\end{figure}

\noindent
If the Kan extension $F = Ran_{K}D$ exists, there must be a unique
natural transformation $\sigma$ from $F'$ to it, such
that $\varepsilon'$ factorizes through $\varepsilon$, that is:
\[\varepsilon' = \varepsilon \cdot (\sigma \circ K)\]
Here, $\sigma \circ K$ is the horizontal composition of two natural
transformations (one of them being the identity natural transformation
on $K$). This transformation is then vertically composed with
$\varepsilon$.

\begin{figure}[H]
  \centering
  \includegraphics[width=0.4\textwidth]{images/kan5.jpg}
\end{figure}

\noindent
In components, when acting on an object $i$ in $\cat{I}$, we get:
\[\varepsilon'_i = \varepsilon_i \circ \sigma_{K i}\]
In our case, $\sigma$ has only one component corresponding to the
single object of $\cat{1}$. So, indeed, this is the unique morphism
from the apex of the cone defined by $F'$ to the apex of
the universal cone defined by $\Ran_{K}D$. The commuting conditions
are exactly the ones required by the definition of a limit.

But, importantly, we are free to replace the trivial category $\cat{1}$
with an arbitrary category $\cat{A}$, and the definition of the right Kan
extension remains valid.

\section{Right Kan Extension}

The right Kan extension of the functor $D \Colon \cat{I} \to \cat{C}$
along the functor $K \Colon \cat{I} \to \cat{A}$ is a functor
$F \Colon \cat{A} \to \cat{C}$ (denoted $\Ran_{K}D$) together with a
natural transformation
\[\varepsilon \Colon F \circ K \to D\]
such that for any other functor $F' \Colon \cat{A} \to \cat{C}$ and
a natural transformation
\[\varepsilon' \Colon F' \circ K \to D\]
there is a unique natural transformation
\[\sigma \Colon F' \to F\]
that factorizes $\varepsilon'$:
\[\varepsilon' = \varepsilon \cdot (\sigma \circ K)\]
This is quite a mouthful, but it can be visualized in this nice diagram:

\begin{figure}[H]
  \centering
  \includegraphics[width=0.4\textwidth]{images/kan7.jpg}
\end{figure}

\noindent
An interesting way of looking at this is to notice that, in a sense, the
Kan extension acts like the inverse of ``functor multiplication.'' Some
authors go as far as use the notation $D/K$ for $\Ran_{K}D$.
Indeed, in this notation, the definition of $\varepsilon$, which is also
called the counit of the right Kan extension, looks like simple
cancellation:
\[\varepsilon \Colon D/K \circ K \to D\]
There is another interpretation of Kan extensions. Consider that the
functor $K$ embeds the category $\cat{I}$ inside $\cat{A}$. In the
simplest case $\cat{I}$ could just be a subcategory of $\cat{A}$. We have
a functor $D$ that maps $\cat{I}$ to $\cat{C}$. Can we extend
$D$ to a functor $F$ that is defined on the whole of
$\cat{A}$? Ideally, such an extension would make the composition
$F \circ K$ be isomorphic to $D$. In other words, $F$
would be extending the domain of $D$ to $\cat{A}$. But a
full-blown isomorphism is usually too much to ask, and we can do with
just half of it, namely a one-way natural transformation $\varepsilon$ from
$F \circ K$ to $D$. (The left Kan extension picks the other direction.)

\begin{figure}[H]
  \centering
  \includegraphics[width=0.4\textwidth]{images/kan6.jpg}
\end{figure}

\noindent
Of course, the embedding picture breaks down when the functor $K$
is not injective on objects or not faithful on hom-sets, as in the
example of the limit. In that case, the Kan extension tries its best to
extrapolate the lost information.

\section{Kan Extension as Adjunction}

Now suppose that the right Kan extension exists for any $D$ (and
a fixed $K$). In that case $\Ran_{K}-$ (with the dash
replacing $D$) is a functor from the functor category
${[}\cat{I}, \cat{C}{]}$ to the functor category ${[}\cat{A}, \cat{C}{]}$. It
turns out that this functor is the right adjoint to the precomposition
functor $- \circ K$. The latter maps functors in ${[}\cat{A}, \cat{C}{]}$
to functors in ${[}\cat{I}, \cat{C}{]}$. The adjunction is:
\[[\cat{I}, \cat{C}](F' \circ K, D) \cong [\cat{A}, \cat{C}](F', \Ran_{K}D)\]
It is just a restatement of the fact that to every natural
transformation we called $\varepsilon'$ corresponds a unique natural
transformation we called $\sigma$.

\begin{figure}[H]
  \centering
  \includegraphics[width=0.4\textwidth]{images/kan92.jpg}
\end{figure}

\noindent
Furthermore, if we chose the category $\cat{I}$ to be the same as
$\cat{C}$, we can substitute the identity functor $I_{\cat{C}}$ for
$D$. We get the following identity:
\[[\cat{C}, \cat{C}](F' \circ K, I_{\cat{C}}) \cong [\cat{A}, \cat{C}](F', \Ran_{K}I_{\cat{C}})\]
We can now chose $F'$ to be the same as $\Ran_{K}I_{\cat{C}}$. In
that case the right hand side contains the identity natural
transformation and, corresponding to it, the left hand side gives us the
following natural transformation:
\[\varepsilon \Colon \Ran_{K}I_{\cat{C}} \circ K \to I_{\cat{C}}\]
This looks very much like the counit of an adjunction:
\[\Ran_{K}I_{\cat{C}} \dashv K\]
Indeed, the right Kan extension of the identity functor along a functor
$K$ can be used to calculate the left adjoint of $K$. For
that, one more condition is necessary: the right Kan extension must be
preserved by the functor $K$. The preservation of the extension
means that, if we calculate the Kan extension of the functor precomposed
with $K$, we should get the same result as precomposing the
original Kan extension with $K$. In our case, this condition
simplifies to:
\[K \circ \Ran_{K}I_{\cat{C}} \cong \Ran_{K}K\]
Notice that, using the division-by-$K$ notation, the adjunction can be
written as:
\[I/K \dashv K\]
which confirms our intuition that an adjunction describes some kind of
an inverse. The preservation condition becomes:
\[K \circ I/K \cong K/K\]
The right Kan extension of a functor along itself, $K/K$, is
called a codensity monad.

The adjunction formula is an important result because, as we'll see
soon, we can calculate Kan extensions using ends (coends), thus giving
us practical means of finding right (and left) adjoints.

\section{Left Kan Extension}

There is a dual construction that gives us the left Kan extension. To
build some intuition, we'll start with the definition of a colimit
and restructure it to use the singleton category $\cat{1}$. We build a
cocone by using the functor $D \Colon \cat{I} \to \cat{C}$ to form its
base, and the functor $F \Colon \cat{1} \to \cat{C}$ to select its apex.

\begin{figure}[H]
  \centering
  \includegraphics[width=0.4\textwidth]{images/kan81.jpg}
\end{figure}

\noindent
The sides of the cocone, the injections, are components of a natural
transformation $\eta$ from $D$ to $F \circ K$.

\begin{figure}[H]
  \centering
  \includegraphics[width=0.4\textwidth]{images/kan10a.jpg}
\end{figure}

\noindent
The colimit is the universal cocone. So for any other functor
$F'$ and a natural transformation
\[\eta' \Colon D \to F' \circ K\]

\begin{figure}[H]
  \centering
  \includegraphics[width=0.4\textwidth]{images/kan10b.jpg}
\end{figure}

\noindent
there is a unique natural transformation $\sigma$ from $F$ to $F'$

\begin{figure}[H]
  \centering
  \includegraphics[width=0.4\textwidth]{images/kan14.jpg}
\end{figure}

\noindent
such that:
\[\eta' = (\sigma \circ K) \cdot \eta\]
This is illustrated in the following diagram:

\begin{figure}[H]
  \centering
  \includegraphics[width=0.4\textwidth]{images/kan112.jpg}
\end{figure}

\noindent
Заменяя синглтонную категорию $\cat{1}$ на $\cat{A}$, это
определение естественно обобщается до определения левого расширения Кана,
обозначаемого $\Lan_{K}D$.

\begin{figure}[H]
  \centering
  \includegraphics[width=0.4\textwidth]{images/kan12.jpg}
\end{figure}

\noindent
Естественное преобразование:
\[\eta \Colon D \to \Lan_{K}D \circ K\]
называется единицей левого расширения Кана.

Как и раньше, мы можем переформулировать взаимно однозначное соответствие между естественными
преобразованиями:
\[\eta' = (\sigma \circ K) \cdot \eta\]
в терминах сопряжения:
\[[\cat{A}, \cat{C}](\Lan_{K}D, F') \cong [\cat{I}, \cat{C}](D, F' \circ K)\]
Другими словами, левое расширение Кана — это левый сопряжённый функтор, а
правое расширение Кана — это правый сопряжённый функтор предкомпозиции с
$K$.

Точно так же, как правое расширение Кана тождественного функтора могло использоваться
для вычисления левого сопряжённого к $K$, левое расширение Кана
тождественного функтора оказывается правым сопряжённым к $K$
(где $\eta$ является единицей сопряжения):
\[K \dashv \Lan_{K}I_{\cat{C}}\]
Объединяя два результата, получаем:
\[\Ran_{K}I_{\cat{C}} \dashv K \dashv \Lan_{K}I_{\cat{C}}\]

\section{Расширения Кана как энды}

Настоящая мощь расширений Кана происходит из того факта, что они могут быть
вычислены с использованием энд (и коэнд). Для простоты мы ограничим наше
внимание случаем, когда целевая категория $\cat{C}$ — это
$\Set$, но формулы можно распространить на любую категорию.

Давайте вернёмся к идее, что расширение Кана может использоваться для расширения
действия функтора за пределы его исходной области определения. Предположим, что
$K$ вкладывает $\cat{I}$ в $\cat{A}$. Функтор $D$ отображает
$\cat{I}$ в $\Set$. Мы могли бы просто сказать, что для любого объекта
$a$ в образе $K$, то есть $a = K i$,
расширенный функтор отображает $a$ в $D i$. Проблема в том, что
делать с теми объектами в $\cat{A}$, которые находятся вне образа
$K$? Идея в том, что каждый такой объект потенциально связан
через множество морфизмов с каждым объектом в образе $K$. Функтор
должен сохранять эти морфизмы. Совокупность морфизмов из
объекта $a$ в образ $K$ характеризуется
hom-функтором:
\[\cat{A}(a, K -)\]

\begin{figure}[H]
  \centering
  \includegraphics[width=0.4\textwidth]{images/kan13.jpg}
\end{figure}

\noindent
Заметим, что этот hom-функтор — это композиция двух функторов:
\[\cat{A}(a, K -) = \cat{A}(a, -) \circ K\]
Правое расширение Кана — это правый сопряжённый функтор композиции:
\[[\cat{I}, \Set](F' \circ K, D) \cong [\cat{A}, \Set](F', \Ran_{K}D)\]
Посмотрим, что произойдёт, если заменить $F'$ на hom-функтор:
\[[\cat{I}, \Set](\cat{A}(a, -) \circ K, D) \cong [\cat{A}, \Set](\cat{A}(a, -), \Ran_{K}D)\]
а затем подставить композицию:
\[[\cat{I}, \Set](\cat{A}(a, K -), D) \cong [\cat{A}, \Set](\cat{A}(a, -), \Ran_{K}D)\]
Правую часть можно упростить с помощью леммы Йонеды:
\[[\cat{I}, \Set](\cat{A}(a, K -), D) \cong \Ran_{K}D a\]
Теперь мы можем переписать множество естественных преобразований как энду, чтобы получить
эту очень удобную формулу для правого расширения Кана:
\[\Ran_{K}D a \cong \int_i \Set(\cat{A}(a, K i), D i)\]
Существует аналогичная формула для левого расширения Кана в терминах
коэнды:
\[\Lan_{K}D a = \int^i \cat{A}(K i, a)\times{}D i\]
Чтобы убедиться, что это так, мы покажем, что это действительно левый
сопряжённый функтор к композиции:
\[[\cat{A}, \Set](\Lan_{K}D, F') \cong [\cat{I}, \Set](D, F' \circ K)\]
Подставим нашу формулу в левую часть:
\[[\cat{A}, \Set](\int^i \cat{A}(K i, -)\times{}D i, F')\]
Это множество естественных преобразований, поэтому его можно переписать как
энду:
\[\int_a \Set(\int^i \cat{A}(K i, a)\times{}D i, F' a)\]
Используя непрерывность hom-функтора, можем заменить коэнду
эндой:
\[\int_a \int_i \Set(\cat{A}(K i, a)\times{}D i, F' a)\]
Можем использовать сопряжение произведение-экспонента:
\[\int_a \int_i \Set(\cat{A}(K i, a),\ (F' a)^{D i})\]
Экспонента изоморфна соответствующему hom-множеству:
\[\int_a \int_i \Set(\cat{A}(K i, a),\ \Set(D i, F' a))\]
Существует теорема, называемая теоремой Фубини, которая позволяет поменять местами
две энды:
\[\int_i \int_a \Set(\cat{A}(K i, a),\ \Set(D i, F' a))\]
Внутренняя энда представляет множество естественных преобразований между двумя
функторами, поэтому можем использовать лемму Йонеды:
\[\int_i \Set(D i, F' (K i))\]
Это действительно множество естественных преобразований, которое образует правую
часть сопряжения, которое мы хотели доказать:
\[[\cat{I}, \Set](D, F' \circ K)\]
Такие вычисления с использованием энд, коэнд и леммы Йонеды
достаточно типичны для ``исчисления'' энд.

\section{Расширения Кана в Haskell}

Формулы энд/коэнд для расширений Кана легко переводятся на
Haskell. Начнём с правого расширения:
\[\Ran_{K}D a \cong \int_i \Set(\cat{A}(a, K i), D i)\]
Заменим энду универсальным квантором, а hom-множества —
типами функций:

\src{snippet01}
Глядя на это определение, ясно, что \code{Ran} должен содержать
значение типа \code{a}, к которому можно применить функцию, и
естественное преобразование между двумя функторами \code{k} и
\code{d}. Например, предположим, что \code{k} — это функтор дерева,
а \code{d} — функтор списка, и вам дан
\code{Ran Tree {[}{]} String}. Если передать ему функцию:

\src{snippet02}
вы получите обратно список \code{Int}, и так далее. Правое расширение
Кана использует вашу функцию для создания дерева, а затем упаковывает его
в список. Например, вы можете передать ему парсер, который генерирует
дерево разбора из строки, и получите список, соответствующий
обходу этого дерева в глубину.

Правое расширение Кана можно использовать для вычисления левого сопряжённого
данного функтора, заменив функтор \code{d} тождественным
функтором. Это приводит к тому, что левый сопряжённый функтора \code{k} представляется
множеством полиморфных функций типа:

\src{snippet03}
Предположим, что \code{k} — это забывающий функтор из категории
моноидов. Тогда универсальный квантор проходит по всем моноидам. Конечно,
в Haskell мы не можем выразить законы моноидов, но следующее является
приличным приближением результирующего свободного функтора (забывающий
функтор \code{k} тождественен на объектах):

\src{snippet04}
Как и ожидалось, это генерирует свободные моноиды, или списки Haskell:

\src{snippet05}
Левое расширение Кана — это коэнда:
\[\Lan_{K}D a = \int^i \cat{A}(K i, a)\times{}D i\]
поэтому оно переводится в экзистенциальный квантор. Символически:

\begin{snip}{haskell}
Lan k d a = exists i. (k i -> a, d i)
\end{snip}
Это можно закодировать в Haskell с помощью \acronym{GADT}ов или используя универсально
квантифицированный конструктор данных:

\src{snippet06}
Интерпретация этой структуры данных заключается в том, что она содержит функцию,
которая берёт контейнер каких-то неуказанных \code{i} и производит
\code{a}. Она также имеет контейнер этих \code{i}. Поскольку вы
не имеете представления, что такое \code{i}, единственное, что можно сделать с этой структурой данных, — это извлечь контейнер \code{i}, упаковать его в
контейнер, определённый функтором \code{k}, используя естественное
преобразование, и вызвать функцию для получения \code{a}. Например,
если \code{d} — дерево, а \code{k} — список, можно
сериализовать дерево, вызвать функцию с результирующим списком и
получить \code{a}.

Левое расширение Кана можно использовать для вычисления правого сопряжённого
функтора. Мы знаем, что правый сопряжённый функтора произведения — это
экспонента, поэтому попробуем реализовать её с помощью расширения Кана:

\src{snippet07}
Это действительно изоморфно типу функции, что подтверждается
следующей парой функций:

\src{snippet08}
Заметим, что, как описано ранее в общем случае, мы выполнили
следующие шаги:

\begin{enumerate}
  \tightlist
  \item
        Извлекли контейнер \code{x} (здесь это
        просто тривиальный тождественный контейнер) и функцию \code{f}.
  \item
        Переупаковали контейнер, используя естественное преобразование между
        тождественным функтором и функтором пары.
  \item
        Вызвали функцию \code{f}.
\end{enumerate}

\section{Свободный функтор}

Интересное приложение расширений Кана — построение
свободного функтора. Это решение следующей практической проблемы:
предположим, у вас есть конструктор типов --- то есть отображение объектов. Возможно ли
определить функтор на основе этого конструктора типов? Другими
словами, можем ли мы определить отображение морфизмов, которое расширит этот конструктор типов до полноценного эндофунктора?

Ключевое наблюдение состоит в том, что конструктор типов можно описать как
функтор, область определения которого — дискретная категория. Дискретная категория не имеет
морфизмов, кроме тождественных морфизмов. Для данной категории $\cat{C}$
мы всегда можем построить дискретную категорию $\cat{|C|}$,
просто отбросив все нетождественные морфизмы. Функтор $F$
из $\cat{|C|}$ в $\cat{C}$ тогда является простым отображением
объектов, или тем, что мы называем конструктором типов в Haskell. Также существует
канонический функтор $J$, который вкладывает $\cat{|C|}$
в $\cat{C}$: он тождественен на объектах (и на тождественных морфизмах).
Левое расширение Кана $F$ вдоль $J$, если оно существует,
тогда является функтором из $\cat{C}$ в $\cat{C}$:
\[\Lan_{J}F a = \int^i \cat{C}(J i, a)\times{}F i\]
Это называется свободным функтором, основанным на $F$.

В Haskell мы бы написали это так:

\src{snippet09}
Действительно, для любого конструктора типов \code{f}, \code{FreeF f} является
функтором:

\src{snippet10}
Как видите, свободный функтор имитирует поднятие функции,
записывая и саму функцию, и её аргумент. Он накапливает поднятые
функции, записывая их композицию. Законы функтора
автоматически выполняются. Это построение использовалось в статье
\urlref{http://okmij.org/ftp/Haskell/extensible/more.pdf}{Freer Monads,
  More Extensible Effects}.

Альтернативно, мы можем использовать правое расширение Кана для той же цели:

\src{snippet11}
Легко проверить, что это действительно функтор:

\src{snippet12}
