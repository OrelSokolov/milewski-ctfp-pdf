% !TEX root = ../../ctfp-print.tex

\lettrine[lhang=0.17]{В}{математике у нас есть} различные способы сказать, что одна вещь похожа на
другую. Самый строгий --- равенство. Две вещи равны, если нет
способа отличить одну от другой. Одна может быть подставлена вместо
другой в каждом мыслимом контексте. Например, замечали ли вы, что мы
использовали \newterm{равенство} морфизмов каждый раз, когда говорили о коммутирующих
диаграммах? Это потому, что морфизмы образуют множество (hom-множество), и элементы множества
могут быть сравнены на равенство.

Но равенство часто слишком сильно. Есть много примеров вещей,
бы одинаковыми для всех практических целей, фактически не являясь
равными. Например, тип пары \code{(Bool, Char)} не
строго равен \code{(Char, Bool)}, но мы понимаем, что они
содержат одну и ту же информацию. Эта концепция лучше всего передаётся
\newterm{изоморфизмом} между двумя типами --- морфизмом, который обратим.
Поскольку это морфизм, он сохраняет структуру; а быть ``изо''
означает, что это часть кругового пути, который приводит вас в то же место, не
важно, с какой стороны вы начинаете. В случае пар этот изоморфизм
называется \code{swap}:

\src{snippet01}
\code{swap} случайно является своим собственным обратным.

\section{Сопряжение и пара единица/коединица}

Когда мы говорим о категориях, являющихся изоморфными, мы выражаем это в терминах
отображений между категориями, т.е. функторов. Мы хотели бы иметь возможность
сказать, что две категории $\cat{C}$ и $\cat{D}$ изоморфны, если
существует функтор $R$ (``правый'') из $\cat{C}$ в $\cat{D}$,
который обратим. Другими словами, существует другой функтор
$L$ (``левый'') из $\cat{D}$ обратно в $\cat{C}$, который, когда
скомпозирован с $R$, равен тождественному функтору $I$.
Есть две возможные композиции, $R \circ L$ и
$L \circ R$; и два возможных тождественных функтора: один в $\cat{C}$
и другой в $\cat{D}$.

\begin{figure}[H]
  \centering
  \includegraphics[width=0.5\textwidth]{images/adj-1.jpg}
\end{figure}

\noindent
Но вот сложная часть: что означает для двух функторов быть
\emph{равными}? Что мы имеем в виду под этим равенством:
\[R \circ L = I_{\cat{D}}\]
или этим:
\[L \circ R = I_{\cat{C}}\]
Было бы разумно определить равенство функторов в терминах равенства
объектов. Два функтора, при действии на равные объекты, должны производить
равные объекты. Но у нас, в общем случае, нет понятия равенства объектов
в произвольной категории. Это просто не часть определения.
(Углубляясь в эту кроличью нору ``что такое равенство на самом деле'', мы
закончили бы в гомотопической теории типов.)

Вы можете возразить, что функторы \emph{являются} морфизмами в категории
категорий, так что они должны быть сравнимы на равенство. И действительно, пока
мы говорим о малых категориях, где объекты образуют множество, мы
можем действительно использовать равенство элементов множества для сравнения
объектов на равенство.

Но помните, $\Cat$ --- это на самом деле $\cat{2}$-категория. Hom-множества в
$\cat{2}$-категории имеют дополнительную структуру --- есть $2$-морфизмы, действующие
между $1$-морфизмами. В $\Cat$ $1$-морфизмы --- функторы, а
$2$-морфизмы --- естественные преобразования. Так что более естественно (не могу
избежать этого каламбура!) рассматривать естественные изоморфизмы как замены
равенству при разговоре о функторах.

Так что вместо изоморфизма категорий имеет смысл рассматривать
более общее понятие \newterm{эквивалентности}. Две категории $\cat{C}$ и
$\cat{D}$ \emph{эквивалентны}, если мы можем найти два функтора, идущие туда и
обратно между ними, чья композиция (в любом направлении)
\newterm{естественно изоморфна} тождественному функтору. Другими словами,
есть двунаправленное естественное преобразование между композицией
$R \circ L$ и тождественным функтором $I_{\cat{D}}$, и другое
между $L \circ R$ и тождественным функтором $I_{\cat{C}}$.

Сопряжение ещё слабее, чем эквивалентность, потому что оно не требует,
чтобы композиция двух функторов была \emph{изоморфна}
тождественному функтору. Вместо этого оно постулирует существование \newterm{однонаправленного}
естественного преобразования из $I_{\cat{D}}$ в $R \circ L$, и
другого из $L \circ R$ в $I_{\cat{C}}$. Вот сигнатуры
этих двух естественных преобразований:
\begin{gather*}
  \eta \Colon I_{\cat{D}} \to R \circ L \\
  \varepsilon \Colon L \circ R \to I_{\cat{C}}
\end{gather*}
$\eta$ называется единицей, а $\varepsilon$ коединицей сопряжения.

Обратите внимание на асимметрию между этими двумя определениями. В общем случае у нас нет
двух оставшихся отображений:
\begin{gather*}
  R \circ L \to I_{\cat{D}} \quad\quad\text{не обязательно} \\
  I_{\cat{C}} \to L \circ R \quad\quad\text{не обязательно}
\end{gather*}
Из-за этой асимметрии функтор $L$ называется
\newterm{левым сопряжённым} к функтору $R$, в то время как функтор
$R$ --- правый сопряжённый к $L$. (Конечно, левый и
правый имеют смысл только если вы рисуете ваши диаграммы одним конкретным способом.)

Компактная нотация для сопряжения:
\[L \dashv R\]
Чтобы лучше понять сопряжение, давайте проанализируем единицу и
коединицу более детально.

\begin{figure}[H]
  \centering
  \includegraphics[width=0.5\textwidth]{images/adj-unit.jpg}
\end{figure}

\noindent
Давайте начнём с единицы. Это естественное преобразование, так что это
семейство морфизмов. Для данного объекта $d$ в $\cat{D}$
компонента $\eta$ --- это морфизм между $I d$, который равен
$d$, и $(R \circ L) d$; который на картинке называется
$d'$:
\[\eta_d \Colon d \to (R \circ L) d\]
Обратите внимание, что композиция $R \circ L$ --- эндофунктор в $\cat{D}$.

Это уравнение говорит нам, что мы можем выбрать любой объект $d$ в
$\cat{D}$ как нашу начальную точку и использовать функтор кругового пути
$R \circ L$, чтобы выбрать наш целевой объект $d'$. Затем мы
пускаем стрелу --- морфизм $\eta_d$ --- в нашу цель.

\begin{figure}[H]
  \centering
  \includegraphics[width=0.5\textwidth]{images/adj-counit.jpg}
\end{figure}

\noindent
По тому же принципу компонента коединицы $\varepsilon$ может быть описана как:
\[\varepsilon_{c} \Colon (L \circ R) c \to c\]
Она говорит нам, что мы
можем выбрать любой объект $c$ в $\cat{C}$ как нашу цель и использовать
функтор кругового пути $L \circ R$, чтобы выбрать источник
$c' = (L \circ R) c$. Затем мы пускаем стрелу --- морфизм
$\varepsilon_{c}$ --- из источника в цель.

Другой способ взглянуть на единицу и коединицу в том, что единица позволяет нам
\emph{ввести} композицию $R \circ L$ везде, где мы могли бы
вставить тождественный функтор на $\cat{D}$; а коединица позволяет нам
\emph{устранить} композицию $L \circ R$, заменяя её
тождеством на $\cat{C}$. Это приводит к некоторым ``очевидным'' условиям согласованности,
которые гарантируют, что введение, за которым следует устранение,
ничего не меняет:
\begin{gather*}
  L = L \circ I_{\cat{D}} \to L \circ R \circ L \to I_{\cat{C}} \circ L = L \\
  R = I_{\cat{D}} \circ R \to R \circ L \circ R \to R \circ I_{\cat{C}} = R
\end{gather*}
Они называются треугольными тождествами, потому что они заставляют следующие
диаграммы коммутировать:

\begin{figure}[H]
  \centering

  \begin{subfigure}
    \centering
    \begin{tikzcd}[column sep=large, row sep=large]
      L \arrow[rd, equal] \arrow[r, "L \circ \eta"]
      & L \circ R \circ L \arrow[d, "\varepsilon \circ L"] \\
      & L
    \end{tikzcd}
  \end{subfigure}%
  \hspace{1cm}
  \begin{subfigure}
    \centering
    \begin{tikzcd}[column sep=large, row sep=large]
      R \arrow[rd, equal] \arrow[r, "\eta \circ R"]
      & R \circ L \circ R \arrow[d, "R \circ \varepsilon"] \\
      & R
    \end{tikzcd}
  \end{subfigure}
\end{figure}

\noindent
Это диаграммы в категории функторов: стрелки --- естественные
преобразования, а их композиция --- горизонтальная композиция
естественных преобразований. В компонентах эти тождества становятся:
\begin{gather*}
  \varepsilon_{L d} \circ L \eta_d = \id_{L d} \\
  R \varepsilon_{c} \circ \eta_{R c} = \id_{R c}
\end{gather*}
Мы часто видим единицу и коединицу в Haskell под разными именами. Единица
известна как \code{return} (или \code{pure}, в определении
\code{Applicative}):

\src{snippet02}
а коединица как \code{extract}:

\src{snippet03}
Здесь \code{m} --- (эндо-) функтор, соответствующий $R \circ L$,
а \code{w} --- (эндо-) функтор, соответствующий $L \circ R$. Как
мы увидим позже, они являются частью определения монады и
комонады, соответственно.

Если вы думаете об эндофункторе как о контейнере, единица (или
\code{return}) --- это полиморфная функция, которая создаёт коробку по умолчанию
вокруг значения произвольного типа. Коединица (или \code{extract}) делает
обратное: она извлекает или производит единственное значение из контейнера.

Мы увидим позже, что каждая пара сопряжённых функторов определяет монаду и
комонаду. Наоборот, каждая монада или комонада может быть факторизована в
пару сопряжённых функторов --- эта факторизация не уникальна, однако.

В Haskell мы используем монады много, но редко факторизуем их в
пары сопряжённых функторов, прежде всего потому, что эти функторы обычно
вывели бы нас из $\Hask$.

Однако мы можем определить сопряжения \newterm{эндофункторов} в Haskell.
Вот часть определения, взятая из
\code{Data.Functor.Adjunction}:

\src{snippet04}
Это определение требует некоторого объяснения. Прежде всего, оно описывает
класс типов с несколькими параметрами --- двумя параметрами являются \code{f} и
\code{u}. Оно устанавливает отношение, называемое \code{Adjunction}, между
этими двумя конструкторами типов.

Дополнительные условия после вертикальной черты специфицируют функциональные
зависимости. Например, \code{f -> u} означает, что
\code{u} определяется \code{f} (отношение между \code{f}
и \code{u} --- функция, здесь на конструкторах типов). Наоборот,
\code{u -> f} означает, что если мы знаем \code{u}, то
\code{f} уникально определён.

Я объясню через мгновение, почему в Haskell мы можем наложить условие,
что правый сопряжённый \code{u} должен быть \newterm{представимым} функтором.

\section{Сопряжения и hom-множества}

Есть эквивалентное определение сопряжения в терминах естественных
изоморфизмов hom-множеств. Это определение хорошо связывается с универсальными
конструкциями, которые мы изучали до сих пор. Каждый раз, когда вы слышите
утверждение, что существует некоторый уникальный морфизм, который факторизует некоторую
конструкцию, вы должны думать об этом как об отображении некоторого множества в
hom-множество. Это значение ``выбора уникального морфизма.''

Кроме того, факторизация часто может быть описана в терминах естественных
преобразований. Факторизация включает коммутирующие диаграммы --- некоторый
морфизм, равный композиции двух морфизмов (факторов).
Естественное преобразование отображает морфизмы в коммутирующие диаграммы. Так что в
универсальной конструкции мы идём от морфизма к коммутирующей диаграмме,
а затем к уникальному морфизму. Мы заканчиваем с отображением из морфизма в
морфизм, или из одного hom-множества в другое (обычно в разных
категориях). Если это отображение обратимо и если оно может быть естественно
расширено на все hom-множества, у нас есть сопряжение.

Главное различие между универсальными конструкциями и сопряжениями в том,
что последние определяются глобально --- для всех hom-множеств. Например,
используя универсальную конструкцию, вы можете определить произведение двух выбранных
объектов, даже если оно не существует для любой другой пары объектов в этой
категории. Как мы скоро увидим, если произведение \emph{любой пары}
объектов существует в категории, оно также может быть определено через
сопряжение.

\begin{figure}[H]
  \centering
  \includegraphics[width=0.5\textwidth]{images/adj-homsets.jpg}
\end{figure}

\noindent
Вот альтернативное определение сопряжения, используя hom-множества. Как
раньше, у нас есть два функтора $L \Colon \cat{D} \to \cat{C}$ и
$R \Colon \cat{C} \to \cat{D}$. Мы выбираем два произвольных объекта:
исходный объект $d$ в $\cat{D}$ и целевой объект $c$
в $\cat{C}$. Мы можем отобразить исходный объект $d$ в $\cat{C}$, используя
$L$. Теперь у нас есть два объекта в $\cat{C}$, $L d$ и
$c$. Они определяют hom-множество:
\[\cat{C}(L d, c)\]
Аналогично, мы можем отобразить целевой объект $c$, используя $R$. Теперь
у нас есть два объекта в $\cat{D}$, $d$ и $R c$. Они
тоже определяют hom-множество:
\[\cat{D}(d, R c)\]
Мы говорим, что $L$ левый сопряжённый к $R$ тогда и только тогда, когда существует
изоморфизм hom-множеств:
\[\cat{C}(L d, c) \cong \cat{D}(d, R c)\]
который естественен как в $d$, так и в $c$.
Естественность означает, что источник $d$ может варьироваться плавно
по $\cat{D}$; а цель $c$ по $\cat{C}$. Более
точно, у нас есть естественное преобразование $\varphi$ между
следующими двумя (ковариантными) функторами из $\cat{C}$ в $\Set$. Вот
действие этих функторов на объекты:
\begin{gather*}
  c \to \cat{C}(L d, c) \\
  c \to \cat{D}(d, R c)
\end{gather*}
Другое естественное преобразование, $\psi$, действует между следующими
(контравариантными) функторами:
\begin{gather*}
  d \to \cat{C}(L d, c) \\
  d \to \cat{D}(d, R c)
\end{gather*}
Оба естественных преобразования должны быть обратимыми.

Легко показать, что два определения сопряжения
эквивалентны. Например, давайте выведем преобразование единицы, начиная
из изоморфизма hom-множеств:
\[\cat{C}(L d, c) \cong \cat{D}(d, R c)\]
Поскольку этот изоморфизм работает для любого объекта $c$, он также должен
работать для $c = L d$:
\[\cat{C}(L d, L d) \cong \cat{D}(d, (R \circ L) d)\]
Мы знаем, что левая сторона должна содержать как минимум один морфизм,
тождество. Естественное преобразование отобразит этот морфизм в
элемент $\cat{D}(d, (R \circ L) d)$ или, вставляя тождественный
функтор $I$, морфизм в:
\[\cat{D}(I d, (R \circ L) d)\]
Мы получаем семейство морфизмов, параметризованных $d$. Они образуют
естественное преобразование между функтором $I$ и функтором
$R \circ L$ (условие естественности легко проверить). Это
в точности наша единица, $\eta$.

Наоборот, начиная с существования единицы и коединицы, мы можем
определить преобразования между hom-множествами. Например, давайте выберем
произвольный морфизм $f$ в hom-множестве $\cat{C}(L d, c)$. Мы
хотим определить $\varphi$, который, действуя на $f$, производит
морфизм в $\cat{D}(d, R c)$.

На самом деле выбора немного. Одна вещь, которую мы можем попробовать, --- поднять
$f$, используя $R$. Это произведёт морфизм $R f$
из $R (L d)$ в $R c$ --- морфизм, который является
элементом $\cat{D}((R \circ L) d, R c)$.

То, что нам нужно для компоненты $\varphi$, --- это морфизм из
$d$ в $R c$. Это не проблема, поскольку мы можем использовать
компоненту $\eta_d$, чтобы попасть из $d$ в
$(R \circ L) d$. Мы получаем:
\[\varphi_f = R f \circ \eta_d\]
Другое направление аналогично, как и вывод $\psi$.

Возвращаясь к определению Haskell для \code{Adjunction}, естественные
преобразования $\varphi$ и $\psi$ заменяются полиморфными
(в \code{a} и \code{b}) функциями \code{leftAdjunct} и
\code{rightAdjunct}, соответственно. Функторы $L$ и
$R$ называются \code{f} и \code{u}:

\src{snippet05}
Эквивалентность между формулировкой \code{unit}/\code{counit}
и формулировкой \code{leftAdjunct}/\allowbreak\code{rightAdjunct}
свидетельствуется этими отображениями:

\src{snippet06}
Очень поучительно следовать за переводом от категорного
описания сопряжения к коду Haskell. Я крайне рекомендую это
как упражнение.

Теперь мы готовы объяснить, почему в Haskell правый сопряжённый
автоматически является \hyperref[representable-functors]{представимым
  функтором}. Причина этого в том, что, в первом приближении, мы
можем рассматривать категорию типов Haskell как категорию множеств.

Когда правая категория $\cat{D}$ --- $\Set$, правый сопряжённый
$R$ --- функтор из $\cat{C}$ в $\Set$. Такой функтор
представим, если мы можем найти объект $\mathit{rep}$ в $\cat{C}$ такой,
что hom-функтор $\cat{C}(\mathit{rep}, \_)$ естественно изоморфен
$R$. Оказывается, что если $R$ --- правый сопряжённый
некоторого функтора $L$ из $\Set$ в $\cat{C}$, такой объект
всегда существует --- это образ одноэлементного множества $()$ под
$L$:
\[\mathit{rep} = L ()\]
Действительно, сопряжение говорит нам, что следующие два hom-множества
естественно изоморфны:
\[\cat{C}(L (), c) \cong \Set((), R c)\]
Для данного $c$ правая сторона --- множество функций из
одноэлементного множества $()$ в $R c$. Мы видели ранее, что
каждая такая функция выбирает один элемент из множества $R c$. Множество
таких функций изоморфно множеству $R c$. Так что у нас есть:
\[\cat{C}(L (), -) \cong R\]
что показывает, что $R$ действительно представим.

\section{Произведение из сопряжения}

Мы ранее вводили несколько концепций, используя универсальные
конструкции. Многие из этих концепций, когда определены глобально, легче
выразить, используя сопряжения. Простейший нетривиальный пример ---
произведение. Суть \hyperref[products-and-coproducts]{универсальной
  конструкции произведения} --- способность факторизовать любого
кандидата, подобного произведению, через универсальное произведение.

Более точно, произведение двух объектов $a$ и $b$ ---
это объект $(a\times{}b)$ (или \code{(a, b)} в нотации Haskell),
оснащённый двумя морфизмами $\mathit{fst}$ и $\mathit{snd}$ такими,
что для любого другого кандидата $c$, оснащённого двумя морфизмами
$p \Colon c \to a$ и $q \Colon c \to b$, существует
уникальный морфизм $m \Colon c \to (a, b)$, который
факторизует $p$ и $q$ через $\mathit{fst}$ и $\mathit{snd}$.

Как мы видели ранее, в Haskell мы можем реализовать \code{factorizer}, который генерирует этот
морфизм из двух проекций:

\src{snippet07}
Легко проверить, что условия факторизации выполняются:

\src{snippet08}
У нас есть отображение, которое принимает пару морфизмов \code{p} и
\code{q} и производит другой морфизм
\code{m = factorizer p q}.

Как мы можем перевести это в отображение между двумя hom-множествами, которые нам
нужны для определения сопряжения? Трюк в том, чтобы выйти за пределы
$\Hask$ и рассматривать пару морфизмов как единственный морфизм в
категории произведения.

Позвольте мне напомнить вам, что такое категория произведения. Возьмите две произвольные
категории $\cat{C}$ и $\cat{D}$. Объекты в категории произведения
$\cat{C}\times{}\cat{D}$ --- пары объектов, один из $\cat{C}$, а другой из
$\cat{D}$. Морфизмы --- пары морфизмов, один из $\cat{C}$, а
один из $\cat{D}$.

Чтобы определить произведение в некоторой категории $\cat{C}$, мы должны начать с
категории произведения $\cat{C}\times{}\cat{C}$. Пары морфизмов из $\cat{C}$ --- единственные
морфизмы в категории произведения $\cat{C}\times{}\cat{C}$.

\begin{figure}[H]
  \centering
  \includegraphics[width=0.5\textwidth]{images/adj-productcat.jpg}
\end{figure}

\noindent
Может быть немного запутанным сначала, что мы используем категорию
произведения для определения произведения. Это, однако, очень разные
произведения. Нам не нужна универсальная конструкция для определения категории
произведения. Всё, что нам нужно, --- это понятие пары объектов и пары
морфизмов.

Однако пара объектов из $\cat{C}$ \emph{не} является объектом в
$\cat{C}$. Это объект в другой категории, $\cat{C}\times{}\cat{C}$. Мы можем
записать пару формально как $\langle a, b \rangle$,
где $a$ и $b$ --- объекты $\cat{C}$. Универсальная
конструкция, с другой стороны, необходима для определения
объекта $a\times{}b$ (или \code{(a, b)} в Haskell), который является объектом
в \emph{той же} категории $\cat{C}$. Этот объект должен
представлять пару $\langle a, b \rangle$ способом,
специфицированным универсальной конструкцией. Он не всегда существует, и
даже если существует для некоторых, может не существовать для других пар объектов
в $\cat{C}$.

Теперь давайте посмотрим на \code{factorizer} как на отображение hom-множеств.
Первое hom-множество находится в категории произведения $\cat{C}\times{}\cat{C}$, а второе ---
в $\cat{C}$. Общий морфизм в $\cat{C}\times{}\cat{C}$ был бы парой
морфизмов $\langle f, g \rangle$:
\begin{gather*}
  f \Colon c' \to a \\
  g \Colon c'' \to b
\end{gather*}
с $c''$ потенциально отличным от
$c'$. Но чтобы определить произведение, мы интересуемся
специальным морфизмом в $\cat{C}\times{}\cat{C}$, парой $p$ и $q$, которые
имеют общий исходный объект $c$. Это нормально: в определении
сопряжения источник левого hom-множества --- не произвольный
объект --- это результат левого функтора $L$, действующего на некоторый
объект из правой категории. Функтор, который подходит, легко
угадать --- это диагональный функтор $\Delta$ из $\cat{C}$ в $\cat{C}\times{}\cat{C}$,
чьё действие на объекты:
\[\Delta c = \langle c, c \rangle\]
Левое hom-множество в нашем сопряжении должно таким образом быть:
\[(\cat{C}\times{}\cat{C})(\Delta c, \langle a, b \rangle)\]
Это hom-множество в категории произведения. Его элементы --- пары
морфизмов, которые мы узнаём как аргументы нашего \code{factorizer}:
\[(c \to a) \to (c \to b) \ldots{}\]
Правое hom-множество живёт в $\cat{C}$, и оно идёт между
исходным объектом $c$ и результатом некоторого функтора $R$,
действующего на целевой объект в $\cat{C}\times{}\cat{C}$. Это функтор, который отображает
пару $\langle a, b \rangle$ в наш объект произведения,
$a\times{}b$. Мы узнаём этот элемент hom-множества как
\emph{результат} \code{factorizer}:
\[\ldots{} \to (c \to (a, b))\]

\begin{figure}[H]
  \centering
  \includegraphics[width=0.5\textwidth]{images/adj-product.jpg}
\end{figure}

\noindent
У нас всё ещё нет полного сопряжения. Для этого нам сначала нужно, чтобы наш
\code{factorizer} был обратим --- мы строим
\emph{изоморфизм} между hom-множествами. Обратный
\code{factorizer} должен начинаться с морфизма $m$ ---
морфизма из некоторого объекта $c$ в объект произведения $a\times{}b$.
Другими словами, $m$ должен быть элементом:
\[\cat{C}(c, a\times{}b)\]
Обратный факторизатор должен отобразить $m$ в морфизм
$\langle p, q \rangle$ в $\cat{C}\times{}\cat{C}$, который идёт из
$\langle c, c \rangle$ в
$\langle a, b \rangle$; другими словами, морфизм,
который является элементом:
\[(\cat{C}\times{}\cat{C})(\Delta\ c, \langle a, b \rangle)\]
Если это отображение существует, мы заключаем, что существует правый сопряжённый
к диагональному функтору. Этот функтор определяет произведение.

В Haskell мы всегда можем сконструировать обратный
\code{factorizer}, композируя \code{m} с, соответственно,
\code{fst} и \code{snd}.

\begin{snip}{haskell}
p = fst . m
q = snd . m
\end{snip}
Чтобы завершить доказательство эквивалентности двух способов определения
произведения, нам также нужно показать, что отображение между hom-множествами
естественно в $a$, $b$ и $c$. Я оставлю это как
упражнение для преданного читателя.

Подводя итог тому, что мы сделали: категорное произведение может быть определено
глобально как \newterm{правый сопряжённый} диагонального функтора:
\[(\cat{C}\times{}\cat{C})(\Delta c, \langle a, b \rangle) \cong \cat{C}(c, a\times{}b)\]
Здесь $a\times{}b$ --- результат действия нашего правого сопряжённого
функтора $\mathit{Product}$ на пару
$\langle a, b \rangle$. Обратите внимание, что любой функтор из
$\cat{C}\times{}\cat{C}$ --- это бифунктор, так что $\mathit{Product}$ --- бифунктор. В
Haskell бифунктор $\mathit{Product}$ записывается просто как
\code{(,)}. Вы можете применить его к двум типам и получить их тип произведения,
например:

\src{snippet09}

\section{Экспоненциал из сопряжения}

Экспоненциал $b^a$, или объект-функция $a \Rightarrow b$, может быть
определён, используя \hyperref[function-types]{универсальную
  конструкцию}. Эта конструкция, если она существует для всех пар объектов,
может рассматриваться как сопряжение. Опять, трюк в том, чтобы сконцентрироваться на
утверждении:

\begin{quote}
  Для любого другого объекта $z$ с морфизмом $g \Colon z\times{}a \to b$
  существует уникальный морфизм $h \Colon z \to (a \Rightarrow b)$
\end{quote}
Это утверждение устанавливает отображение между hom-множествами.

В этом случае мы имеем дело с объектами в одной и той же категории, так что
два сопряжённых функтора --- эндофункторы. Левый (эндо-)функтор
$L$, при действии на объект $z$, производит $z\times{}a$.
Это функтор, который соответствует взятию произведения с некоторым фиксированным
$a$.

Правый (эндо-)функтор $R$, при действии на $b$, производит
объект-функцию $a \Rightarrow b$ (или $b^a$). Опять, $a$
фиксирован. Сопряжение между этими двумя функторами часто записывается как:
\[-\times{}a \dashv (-)^a\]
Отображение hom-множеств, которое лежит в основе этого сопряжения, лучше всего видно,
перерисовав диаграмму, которую мы использовали в универсальной конструкции.

\begin{figure}[H]
  \centering
  \includegraphics[width=0.4\textwidth]{images/adj-expo.jpg}
\end{figure}

\noindent
Обратите внимание, что морфизм $\mathit{eval}$\footnote{См. гл.9 об \hyperref[function-types]{универсальной
    конструкции}.} --- не что иное, как коединица
этого сопряжения:
\[(a \Rightarrow b)\times{}a \to b\]
где:
\[(a \Rightarrow b)\times{}a = (L \circ R) b\]
Я ранее упоминал, что универсальная конструкция определяет
уникальный объект с точностью до изоморфизма. Вот почему у нас есть ``то'' произведение и
``тот'' экспоненциал. Это свойство переводится на сопряжения также: если
функтор имеет сопряжённый, этот сопряжённый уникален с точностью до изоморфизма.

\section{Задачи}

\begin{enumerate}
  \tightlist
  \item
        Выведите квадрат естественности для $\psi$, преобразования
        между двумя (контравариантными) функторами:
        \begin{gather*}
          a \to \cat{C}(L a, b) \\
          a \to \cat{D}(a, R b)
        \end{gather*}
  \item
        Выведите коединицу $\varepsilon$, начиная с изоморфизма hom-множеств во
        втором определении сопряжения.
  \item
        Завершите доказательство эквивалентности двух определений
        сопряжения.
  \item
        Покажите, что копроизведение может быть определено сопряжением. Начните с
        определения факторизатора для копроизведения.
  \item
        Покажите, что копроизведение --- левый сопряжённый диагонального функтора.
  \item
        Определите сопряжение между произведением и объектом-функцией в
        Haskell.
\end{enumerate}
