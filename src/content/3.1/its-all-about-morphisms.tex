% !TEX root = ../../ctfp-print.tex

\lettrine[lhang=0.17]{Е}{сли я ещё не} убедил вас, что теория категорий --- это всё о
морфизмах, то я не выполнил свою работу должным образом. Поскольку следующая тема ---
сопряжения, которые определяются в терминах изоморфизмов hom-множеств,
имеет смысл пересмотреть наши интуиции о строительных блоках
hom-множеств. Также вы увидите, что сопряжения предоставляют более общий
язык для описания многих конструкций, которые мы изучали ранее, так что
может помочь пересмотреть их тоже.

\section{Функторы}

Начнём с того, что вы действительно должны думать о функторах как об отображениях
морфизмов --- взгляд, который подчёркивается в определении Haskell
класса типов \code{Functor}, который вращается вокруг \code{fmap}. Конечно,
функторы также отображают объекты --- конечные точки морфизмов ---
иначе мы не смогли бы говорить о сохранении композиции.
Объекты говорят нам, какие пары морфизмов композиционны. Цель
одного морфизма должна быть равна источнику другого --- если они должны быть
скомпозированы. Так что если мы хотим, чтобы композиция морфизмов отображалась в
композицию \newterm{поднятых} морфизмов, отображение их
конечных точек практически определено.

\section{Коммутирующие диаграммы}

Много свойств морфизмов выражаются в терминах коммутирующих
диаграмм. Если конкретный морфизм может быть описан как композиция
других морфизмов более чем одним способом, то у нас есть коммутирующая диаграмма.

В частности, коммутирующие диаграммы формируют основу почти всех универсальных
конструкций (с заметными исключениями начального и терминального
объектов). Мы видели это в определениях произведений, копроизведений,
различных других (ко-)пределов, экспоненциальных объектов, свободных моноидов и т.д.

Произведение --- это простой пример универсальной конструкции. Мы выбираем два
объекта $a$ и $b$ и смотрим, существует ли объект
$c$ вместе с парой морфизмов $p$ и $q$,
который имеет универсальное свойство быть их произведением.

\begin{figure}[H]
  \centering
  \includegraphics[width=0.3\textwidth]{images/productranking.jpg}
\end{figure}

\noindent
Произведение --- это специальный случай предела. Предел определяется в терминах
конусов. Общий конус построен из коммутирующих диаграмм. Коммутативность
этих диаграмм может быть заменена подходящим условием естественности для
отображения функторов. Таким способом коммутативность сводится к роли
ассемблера для языка более высокого уровня естественных
преобразований.

\section{Естественные преобразования}

В общем случае естественные преобразования очень удобны, когда нам нужно
отображение из морфизмов в коммутирующие квадраты. Две противоположные стороны
квадрата естественности --- это отображения некоторого морфизма $f$ под двумя
функторами $F$ и $G$. Другие стороны --- компоненты
естественного преобразования (которые также являются морфизмами).

\begin{figure}[H]
  \centering
  \includegraphics[width=0.35\textwidth]{images/3_naturality.jpg}
\end{figure}

\noindent
Естественность означает, что когда вы перемещаетесь к ``соседней'' компоненте (под
соседней я имею в виду соединённую морфизмом), вы не идёте против
структуры ни категории, ни функторов. Не важно,
используете ли вы сначала компоненту естественного преобразования, чтобы
перекинуть мост между объектами, а затем прыгаете к её соседу, используя
функтор; или наоборот. Два направления ортогональны.
Естественное преобразование перемещает вас влево и вправо, а функторы перемещают
вас вверх и вниз или вперёд и назад --- так сказать. Вы можете визуализировать
\emph{образ} функтора как лист в целевой категории. Естественное
преобразование отображает один такой лист, соответствующий $F$, в другой,
соответствующий $G$.

\begin{figure}[H]
  \centering
  \includegraphics[width=0.35\textwidth]{images/sheets.png}
\end{figure}

\noindent
Мы видели примеры этой ортогональности в Haskell. Там действие
функтора модифицирует содержимое контейнера без изменения его
формы, в то время как естественное преобразование переупаковывает нетронутое содержимое
в другой контейнер. Порядок этих операций не
важен.

Мы видели конусы в определении предела, заменённые естественными
преобразованиями. Естественность гарантирует, что стороны каждого конуса
коммутируют. Тем не менее, предел определяется в терминах отображений \emph{между}
конусами. Эти отображения также должны удовлетворять условиям коммутативности. (Например,
треугольники в определении произведения должны коммутировать.)

Эти условия тоже могут быть заменены естественностью. Вы можете вспомнить,
что \emph{универсальный} конус, или предел, определяется как естественное
преобразование между (контравариантным) hom-функтором:
\[F \Colon c \to \cat{C}(c, \Lim[D])\]
и (также контравариантным) функтором, который отображает объекты в \emph{C} в
конусы, которые сами являются естественными преобразованиями:
\[G \Colon c \to \mathit{Nat}(\Delta_c, D)\]
Здесь $\Delta_c$ --- постоянный функтор, а $D$ --- функтор,
который определяет диаграмму в $\cat{C}$. Оба функтора $F$ и
$G$ имеют хорошо определённые действия на морфизмы в $\cat{C}$. Так
случается, что это конкретное естественное преобразование между $F$
и $G$ --- \emph{изоморфизм}.

\section{Естественные изоморфизмы}

Естественный изоморфизм --- который является естественным преобразованием, каждая
компонента которого обратима --- это способ теории категорий сказать, что
``две вещи одинаковы.'' Компонента такого преобразования должна
быть изоморфизмом между объектами --- морфизмом, который имеет обратный.
Если вы визуализируете образы функторов как листы, естественный изоморфизм ---
это взаимно однозначное обратимое отображение между этими листами.

\section{Hom-множества}

Но что такое морфизмы? Они имеют больше структуры, чем объекты: в отличие от
объектов, морфизмы имеют два конца. Но если вы фиксируете исходный и
целевой объекты, морфизмы между ними образуют скучное множество (по
крайней мере для локально малых категорий). Мы можем дать элементам этого множества
имена типа $f$ или $g$, чтобы различать один от другого ---
но что же, на самом деле, делает их разными?

Существенное различие между морфизмами в данном hom-множестве заключается в
способе, которым они композируются с другими морфизмами (из смежных hom-множеств). Если
есть морфизм $h$, чья композиция (либо пре-, либо пост-)
с $f$ отличается от таковой с $g$, например:
\[h \circ f \neq h \circ g\]
то мы можем напрямую ``наблюдать'' различие между $f$ и
$g$. Но даже если различие не напрямую наблюдаемо, мы
можем использовать функторы, чтобы увеличить hom-множество. Функтор $F$ может
отобразить два морфизма в различные морфизмы:
\[F f \neq F g\]
в более богатой категории, где смежные hom-множества предоставляют больше
разрешения, например:
\[h' \circ F f \neq h' \circ F g\]
где $h'$ не в образе $F$.

\section{Изоморфизмы hom-множеств}

Много категорных конструкций полагаются на изоморфизмы между
hom-множествами. Но поскольку hom-множества --- это просто множества, простой изоморфизм между
ними не говорит вам много. Для конечных множеств изоморфизм просто говорит,
что они имеют одинаковое количество элементов. Если множества бесконечны,
их кардинальность должна быть одинаковой. Но любой значимый изоморфизм
hom-множеств должен учитывать композицию. А композиция включает
больше, чем одно hom-множество. Нам нужно определить изоморфизмы, которые охватывают целые
коллекции hom-множеств, и нам нужно наложить некоторые условия совместимости,
которые взаимодействуют с композицией. И \newterm{естественный}
изоморфизм точно подходит под это.

Но что такое естественный изоморфизм hom-множеств? Естественность --- это свойство
отображений между функторами, а не множествами. Так что мы действительно говорим о
естественном изоморфизме между функторами со значениями в hom-множествах. Эти функторы ---
больше, чем просто функторы со значениями в множествах. Их действие на морфизмы индуцируется
соответствующими hom-функторами. Морфизмы канонически отображаются
hom-функторами, используя либо пре-, либо пост-композицию (в зависимости от
ковариантности функтора).

Вложение Ёнеды --- один пример такого изоморфизма. Оно отображает
hom-множества в $\cat{C}$ в hom-множества в категории функторов; и оно
естественно. Один функтор во вложении Ёнеды --- это hom-функтор в
$\cat{C}$, а другой отображает объекты в множества естественных преобразований
между hom-множествами.

Определение предела также является естественным изоморфизмом между hom-множествами
(второе, опять, в категории функторов):
\[\cat{C}(c, \Lim[D]) \simeq \mathit{Nat}(\Delta_c, D)\]
Оказывается, что наша конструкция экспоненциального объекта, или
свободного моноида, также может быть переписана как естественный изоморфизм между
hom-множествами.

Это не совпадение --- мы увидим далее, что это просто разные
примеры сопряжений, которые определяются как естественные изоморфизмы
hom-множеств.

\section{Асимметрия hom-множеств}

Есть ещё одно наблюдение, которое поможет нам понять сопряжения.
Hom-множества, в общем случае, не симметричны. Hom-множество $\cat{C}(a, b)$
часто очень отличается от hom-множества $\cat{C}(b, a)$. Окончательная
демонстрация этой асимметрии --- частичный порядок, рассматриваемый как категория.
В частичном порядке морфизм из $a$ в $b$ существует тогда и
только тогда, когда $a$ меньше или равно $b$. Если
$a$ и $b$ различны, то не может быть морфизма,
идущего в другую сторону, из $b$ в $a$. Так что если hom-множество
$\cat{C}(a, b)$ непусто, что в этом случае означает, что это
одноэлементное множество, то $\cat{C}(b, a)$ должно быть пустым, если
$a = b$. Стрелки в этой категории имеют определённый поток в
одном направлении.

Предпорядок, который основан на отношении, которое не обязательно
антисимметрично, также ``в основном'' направленный, за исключением случайных
циклов. Удобно думать о произвольной категории как о
обобщении предпорядка.

Предпорядок --- это тонкая категория --- все hom-множества либо синглетоны, либо
пусты. Мы можем визуализировать общую категорию как ``толстый'' предпорядок.

\section{Задачи}

\begin{enumerate}
  \tightlist
  \item
        Рассмотрите некоторые вырожденные случаи условия естественности и нарисуйте
        соответствующие диаграммы. Например, что происходит, если либо функтор
        $F$, либо $G$ отображает оба объекта $a$ и $b$
        (концы $f \Colon a \to b$) в один и тот же
        объект, например, $F a = F b$ или $G a = G b$?
        (Заметьте, что вы получаете конус или коконус таким образом.) Затем рассмотрите
        случаи, где либо $F a = G a$, либо $F b = G b$.
        Наконец, что если вы начинаете с морфизма, который замыкается сам на себя ---
        $f \Colon a \to a$?
\end{enumerate}
