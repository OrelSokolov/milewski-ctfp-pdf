% !TEX root = ../../ctfp-print.tex

\lettrine[lhang=0.17]{К}{атегория малá}, если её объекты образуют множество. Но мы знаем, что
существуют вещи больше множеств. Известно, что множество всех множеств не может быть
сформировано в рамках стандартной теории множеств (теория Цермело-Френкеля,
опционально дополненная аксиомой выбора). Поэтому категория всех
множеств должна быть большой. Существуют математические трюки, такие как универсумы Гротендика,
которые можно использовать для определения коллекций, выходящих за пределы множеств.
Эти трюки позволяют нам говорить о больших категориях.

Категория является \emph{локально малой}, если морфизмы между любыми двумя объектами
образуют множество. Если они не образуют множество, нам придётся переосмыслить несколько
определений. В частности, что означает композиция морфизмов, если мы
даже не можем выбрать их из множества? Решение заключается в том, чтобы поднять себя за волосы,
заменив hom-множества, которые являются объектами в $\Set$,
\emph{объектами} из некоторой другой категории $\cat{V}$. Разница в том,
что, в общем случае, объекты не имеют элементов, поэтому нам больше не
разрешается говорить об отдельных морфизмах. Мы должны определить все
свойства \emph{обогащённой} категории в терминах операций, которые
могут быть выполнены на hom-объектах в целом. Чтобы сделать это,
категория, которая предоставляет hom-объекты, должна иметь дополнительную структуру --- она
должна быть моноидальной категорией. Если назвать эту моноидальную категорию $\cat{V}$,
мы можем говорить о категории $\cat{C}$, обогащённой над $\cat{V}$.

Помимо соображений размера, нам может быть интересно обобщить hom-множества до
чего-то, что имеет больше структуры, чем просто множества. Например,
традиционная категория не имеет понятия расстояния между
объектами. Два объекта либо соединены морфизмами, либо нет. Все
объекты, соединённые с данным объектом, являются его соседями. В отличие
от реальной жизни, в категории друг друга друга друга так же
близок ко мне, как мой закадычный друг. В подходящей обогащённой категории мы можем
определить расстояния между объектами.

Есть ещё одна очень практическая причина получить некоторый опыт с
обогащёнными категориями — это то, что очень полезный онлайн-источник
категориальных знаний, \urlref{https://ncatlab.org/}{nLab}, написан
в основном в терминах обогащённых категорий.

\section{Почему моноидальная категория?}

При построении обогащённой категории мы должны помнить, что
нужно иметь возможность восстановить обычные определения, когда заменим
моноидальную категорию на $\Set$, а hom-объекты на hom-множества.
Лучший способ достичь этого — начать с обычных определений и
продолжать переформулировать их бесточечным способом --- то есть, не
называя элементы множеств.

Начнём с определения композиции. Обычно она берёт
пару морфизмов, один из $\cat{C}(b, c)$ и один из
$\cat{C}(a, b)$, и отображает их в морфизм из $\cat{C}(a, c)$. Другими
словами, это отображение:
\[\cat{C}(b, c)\times{}\cat{C}(a, b) \to \cat{C}(a, c)\]
Это функция между множествами --- одно из них является декартовым
произведением двух hom-множеств. Эту формулу можно легко обобщить,
заменив декартово произведение чем-то более общим. Категориальное
произведение подойдёт, но можем пойти ещё дальше и использовать совершенно
общее тензорное произведение.

Далее идут тождественные морфизмы. Вместо выбора отдельных элементов
из hom-множеств, можем определить их, используя функции из синглтонного множества
$\cat{1}$:
\[j_a \Colon \cat{1} \to \cat{C}(a, a)\]
Опять же, можем заменить синглтонное множество терминальным объектом, но
можем пойти ещё дальше, заменив его единицей $i$ тензорного
произведения.

Как видите, объекты, взятые из некоторой моноидальной категории $\cat{V}$, являются
хорошими кандидатами на замену hom-множеств.

\section{Моноидальная категория}

Мы уже говорили о моноидальных категориях раньше, но стоит повторить
определение. Моноидальная категория определяет тензорное произведение, которое является
бифунктором:
\[\otimes \Colon \cat{V}\times{}\cat{V} \to \cat{V}\]
Мы хотим, чтобы тензорное произведение было ассоциативным, но достаточно выполнения
ассоциативности с точностью до естественного изоморфизма. Этот изоморфизм называется
ассоциатором. Его компоненты:
\[\alpha_{a b c} \Colon (a \otimes b) \otimes c \to a \otimes (b \otimes c)\]
Он должен быть естественным по всем трём аргументам.

Моноидальная категория также должна определять специальный единичный объект $i$,
который служит единицей тензорного произведения; опять же, с точностью до естественного
изоморфизма. Два изоморфизма называются, соответственно, левым и
правым унитором, и их компоненты:
\begin{align*}
  \lambda_a & \Colon i \otimes a \to a \\
  \rho_a    & \Colon a \otimes i \to a
\end{align*}
Ассоциатор и униторы должны удовлетворять условиям когерентности:

\begin{figure}[H]
  \centering
  \begin{tikzcd}[row sep=large]
    ((a \otimes b) \otimes c) \otimes d
    \arrow[d, "\alpha_{(a \otimes b)cd}"]
    \arrow[rr, "\alpha_{abc} \otimes \id_d"]
    & & (a \otimes (b \otimes c)) \otimes d
    \arrow[d, "\alpha_{a(b \otimes c)d}"] \\
    (a \otimes b) \otimes (c \otimes d)
    \arrow[rd, "\alpha_{ab(c \otimes d)}"]
    & & a \otimes ((b \otimes c) \otimes d)
    \arrow[ld, "\id_a \otimes \alpha_{bcd}"] \\
    & a \otimes (b \otimes (c \otimes d))
  \end{tikzcd}
\end{figure}

\begin{figure}[H]
  \centering
  \begin{tikzcd}[row sep=large]
    (a \otimes i) \otimes b
    \arrow[dr, "\rho_{a} \otimes \id_b"']
    \arrow[rr, "\alpha_{aib}"]
    & & a \otimes (i \otimes b)
    \arrow[dl, "\id_a \otimes \lambda_b"] \\
    & a \otimes b
  \end{tikzcd}
\end{figure}

\noindent
Моноидальная категория называется \newterm{симметричной}, если существует естественный
изоморфизм с компонентами:
\[\gamma_{a b} \Colon a \otimes b \to b \otimes a\]
чей ``квадрат равен единице'':
\[\gamma_{b a} \circ \gamma_{a b} = \idarrow[a \otimes b]\]
и который согласован с моноидальной структурой.

Интересная вещь в моноидальных категориях — это то, что можно
определить внутренний hom (объект функции) как правый сопряжённый к
тензорному произведению. Вы можете вспомнить, что стандартное определение
объекта функции, или экспоненты, было через правый сопряжённый к
категориальному произведению. Категория, в которой такой объект существовал для
любой пары объектов, называлась декартово замкнутой. Вот сопряжение,
которое определяет внутренний hom в моноидальной категории:
\[\cat{V}(a \otimes b, c) \sim \cat{V}(a, [b, c])\]
Следуя
\urlref{http://www.tac.mta.ca/tac/reprints/articles/10/tr10.pdf}{G. M.
  Kelly}, я использую обозначение ${[}b, c{]}$ для внутреннего
hom. Коединица этого сопряжения — это естественное преобразование, чьи
компоненты называются морфизмами вычисления:
\[\varepsilon_{a b} \Colon ([a, b] \otimes a) \to b\]
Заметим, что если тензорное произведение не симметрично, можем определить
другой внутренний hom, обозначаемый ${[}{[}a, c{]}{]}$, используя
следующее сопряжение:
\[\cat{V}(a \otimes b, c) \sim \cat{V}(b, [[a, c]])\]
Моноидальная категория, в которой оба определены, называется \newterm{бизамкнутой}. Пример
категории, которая не бизамкнута, — это категория
эндофункторов в $\Set$, где композиция функторов служит тензорным
произведением. Это категория, которую мы использовали для определения монад.

\section{Обогащённая категория}

Категория $\cat{C}$, обогащённая над моноидальной категорией $\cat{V}$, заменяет
hom-множества hom-объектами. Каждой паре объектов $a$ и
$b$ в $\cat{C}$ мы сопоставляем объект $\cat{C}(a, b)$ в
$\cat{V}$. Мы используем то же обозначение для hom-объектов, что и для
hom-множеств, с пониманием, что они не содержат морфизмов. С
другой стороны, $\cat{V}$ — это обычная (необогащённая) категория с
hom-множествами и морфизмами. Так что мы не полностью избавились от множеств --- мы просто
спрятали их под ковёр.

Поскольку мы не можем говорить об отдельных морфизмах в $\cat{C}$, композиция
морфизмов заменяется семейством морфизмов в $\cat{V}$:
\[\circ \Colon \cat{C}(b, c) \otimes \cat{C}(a, b) \to \cat{C}(a, c)\]

\begin{figure}[H]
  \centering
  \includegraphics[width=0.45\textwidth]{images/composition.jpg}
\end{figure}

\noindent
Аналогично, тождественные морфизмы заменяются семейством морфизмов в
$\cat{V}$:
\[j_a \Colon i \to \cat{C}(a, a)\]
где $i$ — это единица тензорного произведения в $\cat{V}$.

\begin{figure}[H]
  \centering
  \includegraphics[width=0.4\textwidth]{images/id.jpg}
\end{figure}

\noindent
Ассоциативность композиции определяется в терминах ассоциатора в
$\cat{V}$:

\begin{figure}[H]
  \centering
  \begin{tikzcd}[column sep=large]
    (\cat{C}(c,d) \otimes \cat{C}(b,c)) \otimes \cat{C}(a,b)
    \arrow[r, "\circ\otimes\id"]
    \arrow[dd, "\alpha"]
    & \cat{C}(b,d) \otimes \cat{C}(a,b)
    \arrow[dr, "\circ"] \\
    & & \cat{C}(a,d) \\
    \cat{C}(c,d) \otimes (\cat{C}(b,c) \otimes \cat{C}(a,b))
    \arrow[r, "\id\otimes\circ"]
    & \cat{C}(c,d) \otimes \cat{C}(a,c)
    \arrow[ur, "\circ"]
  \end{tikzcd}
\end{figure}

\noindent
Unit laws are likewise expressed in terms of unitors:

\begin{figure}[H]
  \centering
  \begin{subfigure}
    \centering
    \begin{tikzcd}[row sep=large]
      \cat{C}(a,b) \otimes i
      \arrow[rr, "\id \otimes j_a"]
      \arrow[dr, "\rho"]
      & & \cat{C}(a,b) \otimes \cat{C}(a,a)
      \arrow[dl, "\circ"] \\
      & \cat{C}(a,b)
    \end{tikzcd}
  \end{subfigure}
  \hspace{1cm}
  \begin{subfigure}
    \centering
    \begin{tikzcd}[row sep=large]
      i \otimes \cat{C}(a,b)
      \arrow[rr, "j_b \otimes \id"]
      \arrow[dr, "\lambda"]
      & & \cat{C}(b,b) \otimes \cat{C}(a,b)
      \arrow[dl, "\circ"] \\
      & \cat{C}(a,b)
    \end{tikzcd}
  \end{subfigure}
\end{figure}

\section{Предпорядки}

Предпорядок определяется как тонкая категория, в которой каждое hom-множество
либо пусто, либо является синглтоном. Мы интерпретируем непустое множество
$\cat{C}(a, b)$ как доказательство того, что $a$ меньше или равно
$b$. Такая категория может интерпретироваться как обогащённая над очень
простой моноидальной категорией, содержащей всего два объекта, $0$ и $1$
(иногда называемые $\mathit{False}$ и $\mathit{True}$). Помимо обязательных тождественных
морфизмов, эта категория имеет единственный морфизм, идущий из $0$ в $1$, назовём
его $0 \to 1$. В ней может быть установлена простая моноидальная структура
с тензорным произведением, моделирующим простую
арифметику $0$ и $1$ (т.е., единственное ненулевое произведение — это $1 \otimes 1$).
Единичный объект в этой категории — $1$. Это строгая моноидальная
категория, то есть ассоциатор и униторы являются тождественными
морфизмами.

Поскольку в предпорядке hom-множество либо пусто, либо является синглтоном, мы можем
легко заменить его hom-объектом из нашей крошечной категории. Обогащённый
предпорядок $\cat{C}$ имеет hom-объект $\cat{C}(a, b)$ для любой пары
объектов $a$ и $b$. Если $a$ меньше или равно
$b$, этот объект равен $1$; иначе он равен $0$.

Давайте рассмотрим композицию. Тензорное произведение любых двух объектов
равно $0$, если только оба из них не равны $1$, в этом случае оно равно $1$. Если это $0$, то
у нас есть два варианта для морфизма композиции: это может быть либо
$\idarrow[0]$, либо $0 \to 1$. Но если это $1$, то единственный
вариант — $\idarrow[1]$. Переводя это обратно к отношениям, это говорит,
что если $a \leqslant b$ и $b \leqslant c$, то
$a \leqslant c$, что в точности является законом транзитивности, который нам
нужен.

Как насчёт тождества? Это морфизм из $1$ в $\cat{C}(a, a)$.
Есть только один морфизм, идущий из $1$, и это тождественный морфизм
$\idarrow[1]$, поэтому $\cat{C}(a, a)$ должен быть $1$. Это означает, что
$a \leqslant a$, что является законом рефлексивности для
предпорядка. Таким образом, и транзитивность, и рефлексивность автоматически
выполняются, если реализовать предпорядок как обогащённую категорию.

\section{Метрические пространства}

Интересный пример принадлежит
\urlref{http://www.tac.mta.ca/tac/reprints/articles/1/tr1.pdf}{Уильяму
  Ловеру}. Он заметил, что метрические пространства можно определить с помощью обогащённых
категорий. Метрическое пространство определяет расстояние между любыми двумя объектами.
Это расстояние — неотрицательное действительное число. Удобно включить
бесконечность как возможное значение. Если расстояние бесконечно, нет
способа добраться от начального объекта до целевого объекта.

Есть некоторые очевидные свойства, которые должны удовлетворяться
расстояниями. Одно из них — то, что расстояние от объекта до себя
должно быть нулём. Другое — неравенство треугольника: прямое расстояние
не больше суммы расстояний с промежуточными остановками. Мы не
требуем, чтобы расстояние было симметричным, что может показаться странным сначала,
но, как объяснил Ловере, вы можете представить, что в одном направлении вы
идёте в гору, а в другом спускаетесь вниз. В любом случае,
симметрия может быть наложена позже как дополнительное ограничение.

Итак, как метрическое пространство можно представить в категориальном языке? Мы должны
построить категорию, в которой hom-объектами являются расстояния. Учтите,
расстояния — это не морфизмы, а hom-объекты. Как hom-объект может быть
числом? Только если мы можем построить моноидальную категорию $\cat{V}$, в которой
эти числа являются объектами. Неотрицательные действительные числа (плюс бесконечность)
образуют полный порядок, поэтому их можно рассматривать как тонкую категорию.
Морфизм между двумя такими числами $x$ и $y$ существует тогда и
только тогда, когда $x \geqslant y$ (примечание: это противоположное
направление тому, которое традиционно используется в определении
предпорядка). Моноидальная структура задаётся сложением, причём ноль
служит единичным объектом. Другими словами, тензорное произведение двух
чисел — это их сумма.

Метрическое пространство — это категория, обогащённая над такой моноидальной категорией.
hom-объект $\cat{C}(a, b)$ из объекта $a$ в $b$ — это
неотрицательное (возможно, бесконечное) число, которое мы будем называть расстоянием
от $a$ до $b$. Посмотрим, что мы получаем для тождества и
композиции в такой категории.

По нашим определениям, морфизм из тензорной единицы, которая является
числом ноль, в hom-объект $\cat{C}(a, a)$ — это отношение:
\[0 \geqslant \cat{C}(a, a)\]
Поскольку $\cat{C}(a, a)$ — неотрицательное число, это условие говорит
нам, что расстояние от $a$ до $a$ всегда равно нулю.
Проверено!

Теперь поговорим о композиции. Начнём с тензорного произведения
двух смежных hom-объектов, $\cat{C}(b, c) \otimes \cat{C}(a, b)$. Мы определили
тензорное произведение как сумму двух расстояний. Композиция — это
морфизм в $\cat{V}$ из этого произведения в $\cat{C}(a, c)$. Морфизм
в $\cat{V}$ определяется как отношение больше-или-равно. Другими словами,
сумма расстояний от $a$ до $b$ и от $b$
до $c$ больше или равна расстоянию от $a$
до $c$. Но это просто стандартное неравенство треугольника. Проверено!

Переформулировав метрическое пространство в терминах обогащённой категории, мы получаем
неравенство треугольника и нулевое расстояние до себя ``бесплатно''.

\section{Обогащённые функторы}

Определение функтора включает отображение морфизмов. В
обогащённом контексте у нас нет понятия отдельных морфизмов, поэтому
мы должны иметь дело с hom-объектами целиком. Hom-объекты — это объекты в
моноидальной категории $\cat{V}$, и у нас есть морфизмы между ними в нашем
распоряжении. Поэтому имеет смысл определить обогащённые функторы между
категориями, когда они обогащены над одной и той же моноидальной категорией
$\cat{V}$. Тогда мы можем использовать морфизмы в $\cat{V}$ для отображения hom-объектов
между двумя обогащёнными категориями.

\newterm{Обогащённый функтор} $F$ между двумя категориями $\cat{C}$
и $\cat{D}$, помимо отображения объектов в объекты, также приписывает каждой
паре объектов в $\cat{C}$ морфизм в $\cat{V}$:
\[F_{a b} \Colon \cat{C}(a, b) \to \cat{D}(F a, F b)\]
Функтор — это структуросохраняющее отображение. Для обычных функторов это
означало сохранение композиции и тождества. В обогащённом контексте
сохранение композиции означает, что следующая диаграмма коммутирует:

\begin{figure}[H]
  \centering
  \begin{tikzcd}[column sep=large, row sep=large]
    \cat{C}(b,c) \otimes \cat{C}(a,b)
    \arrow[r, "\circ"]
    \arrow[d, "F_{bc} \otimes F_{ab}"]
    & \cat{C}(a,c)
    \arrow[d, "F_{ac}"] \\
    \cat{D}(F b, F c) \otimes \cat{D}(F a, F b)
    \arrow[r,  "\circ"]
    & \cat{D}(F a, F c)
  \end{tikzcd}
\end{figure}

\noindent
Сохранение тождества заменяется сохранением
морфизмов в $\cat{V}$, которые ``выбирают'' тождество:

\begin{figure}[H]
  \centering
  \begin{tikzcd}[row sep=large]
    & i \arrow[dl, "j_a"'] \arrow[dr, "j_{F a}"] & \\
    \cat{C}(a,a)
    \arrow[rr, "F_{aa}"]
    & & \cat{D}(F a, F a)
  \end{tikzcd}
\end{figure}

\section{Самообогащение}

Замкнутая симметричная моноидальная категория может быть самообогащена путём замены
hom-множеств внутренними hom (см. определение выше). Чтобы это
работало, мы должны определить закон композиции для внутренних hom. Другими
словами, мы должны реализовать морфизм со следующей сигнатурой:
\[[b, c] \otimes [a, b] \to [a, c]\]
Это не сильно отличается от любой другой задачи программирования, за исключением того, что
в теории категорий мы обычно используем бесточечные реализации. Начнём
с указания множества, элементом которого это должно быть. В данном случае
это элемент hom-множества:
\[\cat{V}([b, c] \otimes [a, b], [a, c])\]
Это hom-множество изоморфно:
\[\cat{V}(([b, c] \otimes [a, b]) \otimes a, c)\]
Я только что использовал сопряжение, которое определяло внутренний hom
${[}a, c{]}$. Если мы можем построить морфизм в этом новом множестве,
сопряжение укажет нам на морфизм в исходном множестве, который мы
затем можем использовать как композицию. Мы строим этот морфизм, композируя
несколько морфизмов, которые есть в нашем распоряжении. Для начала можем использовать
ассоциатор $\alpha_{{[}b, c{]}\ {[}a, b{]}\ a}$, чтобы переассоциировать
выражение слева:
\[([b, c] \otimes [a, b]) \otimes a \to [b, c] \otimes ([a, b] \otimes a)\]
Можем следовать за ним с коединицей сопряжения $\varepsilon_{a b}$:
\[[b, c] \otimes ([a, b] \otimes a) \to [b, c] \otimes b\]
И использовать коединицу $\varepsilon_{b c}$ снова, чтобы добраться до $c$. Мы, таким образом,
построили морфизм:
\[\varepsilon_{b c} \circ (\idarrow[{[b, c]}] \otimes \varepsilon_{a b}) \circ \alpha_{[b, c] [a, b] a}\]
который является элементом hom-множества:
\[\cat{V}(([b, c] \otimes [a, b]) \otimes a, c)\]
Сопряжение даст нам закон композиции, который мы искали.

Аналогично, тождество:
\[j_a \Colon i \to [a, a]\]
является элементом следующего hom-множества:
\[\cat{V}(i, [a, a])\]
которое изоморфно, через сопряжение, к:
\[\cat{V}(i \otimes a, a)\]
Мы знаем, что это hom-множество содержит левую единицу $\lambda_a$. Можем
определить $j_a$ как её образ при сопряжении.

Практический пример самообогащения — категория $\Set$, которая
служит прототипом для типов в языках программирования. Мы видели
раньше, что это замкнутая моноидальная категория относительно декартова
произведения. В $\Set$ hom-множество между любыми двумя множествами само является
множеством, поэтому это объект в $\Set$. Мы знаем, что оно изоморфно
экспоненциальному множеству, поэтому внешние и внутренние hom
эквивалентны. Теперь мы также знаем, что через самообогащение можно использовать
экспоненциальное множество как hom-объект и выразить композицию в терминах
декартовых произведений экспоненциальных объектов.

\section{Связь с $\cat{2}$-категориями}

Я говорил о $\cat{2}$-категориях в контексте $\Cat$, категории
(малых) категорий. Морфизмы между категориями — это функторы,
но есть дополнительная структура: естественные преобразования между
функторами. В $\cat{2}$-категории объекты часто называются нульмерными ячейками;
морфизмы — $1$-ячейками; а морфизмы между морфизмами — $2$-ячейками. В
$\Cat$ $0$-ячейки — это категории, $1$-ячейки — функторы, а
$2$-ячейки — естественные преобразования.

Но заметим, что функторы между двумя категориями также образуют категорию; поэтому
в $\Cat$ у нас действительно есть \emph{hom-категория}, а не
hom-множество. Оказывается, что так же, как $\Set$ можно рассматривать как
категорию, обогащённую над $\Set$, $\Cat$ можно рассматривать как
категорию, обогащённую над $\Cat$. Ещё более общо, так же, как
каждая категория может рассматриваться как обогащённая над $\Set$, каждая
$\cat{2}$-категория может считаться обогащённой над $\Cat$.
