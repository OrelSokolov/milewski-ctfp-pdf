% !TEX root = ../../ctfp-print.tex

\lettrine[lhang=0.17]{Н}{ет хорошего места}, чтобы закончить книгу о теории категорий. Всегда есть
что изучать. Теория категорий — обширный предмет. В то же время очевидно,
что одни и те же темы, концепции и паттерны продолжают появляться
снова и снова. Есть поговорка, что все концепции — это расширения Кана
и, действительно, можно использовать расширения Кана, чтобы вывести пределы,
копределы, сопряжения, монады, лемму Йонеды и многое другое.
Понятие самой категории возникает на всех уровнях абстракции, как и
концепция моноида и монады. Какая из них самая базовая?
Оказывается, они все взаимосвязаны, одна ведёт к другой в
бесконечном цикле абстракций. Я решил, что показать эти
взаимосвязи может быть хорошим способом закончить эту книгу.

\section{Бикатегории}

Один из самых сложных аспектов теории категорий — постоянная
смена перспектив. Возьмём категорию множеств, например. Мы
привыкли определять множества в терминах элементов. Пустое множество не имеет
элементов. Синглтонное множество имеет один элемент. Декартово произведение двух
множеств — это множество пар, и так далее. Но когда говорили о категории
$\Set$, я просил вас забыть о содержимом множеств и
вместо этого сконцентрироваться на морфизмах (стрелках) между ними. Вам было
позволено время от времени заглядывать под капот, чтобы увидеть, что
конкретная универсальная конструкция в $\Set$ описывает в терминах
элементов. Терминальный объект оказался множеством с одним элементом,
и так далее. Но это были просто проверки на вменяемость.

Функтор определяется как отображение категорий. Естественно
рассматривать отображение как морфизм в категории. Функтор оказался
морфизмом в категории категорий (малых категорий, если мы
хотим избежать вопросов о размере). Рассматривая функтор как стрелку,
мы теряем информацию о его действии на внутренности
категории (её объекты и морфизмы), так же как теряем
информацию о действии функции на элементы множества, когда
рассматриваем её как стрелку в $\Set$. Но функторы между любыми двумя
категориями также образуют категорию. На этот раз вас просят рассмотреть
что-то, что было стрелкой в одной категории, как объект в другой.
В категории функторов функторы — это объекты, а естественные преобразования —
морфизмы. Мы обнаружили, что одна и та же вещь может быть стрелкой в
одной категории и объектом в другой. Наивный взгляд на объекты как на
существительные, а на стрелки как на глаголы не работает.

Вместо переключения между двумя взглядами, можем попытаться объединить их в
один. Так мы получаем концепцию $\cat{2}$-категории, в которой объекты
называются $0$-ячейками, морфизмы — $1$-ячейками, а морфизмы между
морфизмами — $2$-ячейками.

\begin{figure}[H]
  \centering
  \includegraphics[width=0.35\textwidth]{images/twocat.png}
  \caption{$0$-ячейки $a, b$; $1$-ячейки $f, g$; и $2$-ячейка $\alpha$.}
\end{figure}

\noindent
Категория категорий $\Cat$ — непосредственный пример. У нас есть
категории как $0$-ячейки, функторы как $1$-ячейки и естественные преобразования
как $2$-ячейки. Законы $\cat{2}$-категории говорят нам, что $1$-ячейки между любыми
двумя $0$-ячейками образуют категорию (другими словами, $\cat{C}(a, b)$ —
hom-категория, а не hom-множество). Это хорошо согласуется с нашим более ранним
утверждением, что функторы между любыми двумя категориями образуют категорию функторов.

В частности, $1$-ячейки из любой $0$-ячейки обратно к себе также образуют
категорию, hom-категорию $\cat{C}(a, a)$; но эта категория имеет ещё
больше структуры. Члены $\cat{C}(a, a)$ можно рассматривать как стрелки в
$\cat{C}$ или как объекты в $\cat{C}(a, a)$. Как стрелки, они могут быть
композированы друг с другом. Но когда смотрим на них как на объекты,
композиция становится отображением из пары объектов в объект. На
самом деле это очень похоже на произведение --- точнее, на тензорное произведение. Это тензорное произведение имеет единицу: тождественную $1$-ячейку. Оказывается,
что в любой $\cat{2}$-категории hom-категория $\cat{C}(a, a)$
автоматически является моноидальной категорией с тензорным произведением, определённым как
композиция $1$-ячеек. Законы ассоциативности и единицы просто следуют из
соответствующих законов категории.

Посмотрим, что это означает в нашем каноническом примере $\cat{2}$-категории
$\Cat$. hom-категория $\Cat(a, a)$ — это категория
эндофункторов на $a$. Композиция эндофункторов играет роль
тензорного произведения в ней. Тождественный функтор — единица относительно
этого произведения. Мы видели ранее, что эндофункторы образуют моноидальную
категорию (мы использовали этот факт в определении монады), но теперь
видим, что это более общее явление: эндо-$1$-ячейки в любой
$\cat{2}$-категории образуют моноидальную категорию. Мы вернёмся к этому позже, когда будем
обобщать монады.

Вы можете вспомнить, что в общей моноидальной категории мы не настаивали
на том, чтобы законы моноида выполнялись точно. Часто было достаточно, чтобы
законы единицы и ассоциативности выполнялись с точностью до
изоморфизма. В $\cat{2}$-категории моноидальные законы в $\cat{C}(a, a)$ следуют
из законов композиции для $1$-ячеек. Эти законы строгие, поэтому мы всегда
получим строгую моноидальную категорию. Однако возможно ослабить
и эти законы. Можем сказать, например, что композиция
тождественной $1$-ячейки $\idarrow[a]$ с другой $1$-ячейкой,
$f \Colon a \to b$, изоморфна, а не равна,
$f$. Изоморфизм $1$-ячеек определяется с использованием $2$-ячеек. Другими
словами, существует $2$-ячейка:
\[\rho \Colon f \circ \idarrow[a] \to f\]
которая имеет обратную.

\begin{figure}[H]
  \centering
  \includegraphics[width=0.35\textwidth]{images/bicat.png}
  \caption{Закон единицы в бикатегории выполняется с точностью до изоморфизма (обратимая
    $2$-ячейка $\rho$).}
\end{figure}

\noindent
Можем сделать то же самое для левой единицы и законов ассоциативности. Этот
вид ослабленной $\cat{2}$-категории называется бикатегорией (есть некоторые
дополнительные законы когерентности, которые я опущу здесь).

Как и ожидалось, эндо-$1$-ячейки в бикатегории образуют общую моноидальную
категорию с нестрогими законами.

Интересный пример бикатегории — категория спанов. Спан
между двумя объектами $a$ и $b$ — это объект $x$
и пара морфизмов:
\begin{gather*}
  f \Colon x \to a \\
  g \Colon x \to b
\end{gather*}

\begin{figure}[H]
  \centering
  \includegraphics[width=0.35\textwidth]{images/span.png}
\end{figure}

\noindent
Вы можете вспомнить, что мы использовали спаны в определении категориального
произведения. Здесь мы хотим рассмотреть спаны как $1$-ячейки в бикатегории.
Первый шаг — определить композицию спанов. Предположим, что у нас есть
примыкающий спан:
\begin{gather*}
  f' \Colon y \to b \\
  g' \Colon y \to c
\end{gather*}

\begin{figure}[H]
  \centering
  \includegraphics[width=0.5\textwidth]{images/compspan.png}
\end{figure}

\noindent
Композиция была бы третьим спаном с некоторой вершиной $z$. Самый
естественный выбор для неё — откат $g$ вдоль
$f'$. Помните, что откат — это объект $z$
вместе с двумя морфизмами:
\begin{align*}
  h  & \Colon z \to x \\
  h' & \Colon z \to y
\end{align*}
такими, что:
\[g \circ h = f' \circ h'\]
что универсально среди всех таких объектов.

\begin{figure}[H]
  \centering
  \includegraphics[width=0.5\textwidth]{images/pullspan.png}
\end{figure}

\noindent
Пока давайте сконцентрируемся на спанах над категорией множеств. В этом
случае откат — это просто множество пар $(p, q)$ из
декартова произведения $x \times y$ таких, что:
\[g\ p = f'\ q\]
Морфизм между двумя спанами, имеющими одинаковые концы, определяется как
морфизм $h$ между их вершинами такой, что соответствующие
треугольники коммутируют.

\begin{figure}[H]
  \centering
  \includegraphics[width=0.4\textwidth]{images/morphspan.png}
  \caption{$2$-ячейка в $\cat{Span}$.}
\end{figure}

\noindent
Подведём итог: в бикатегории $\cat{Span}$ $0$-ячейки — множества, $1$-ячейки —
спаны, $2$-ячейки — морфизмы спанов. Тождественная $1$-ячейка — это
вырожденный спан, в котором все три объекта одинаковы, а два
морфизма — тождественные.

Мы видели другой пример бикатегории раньше: бикатегорию
$\cat{Prof}$
\hyperref[ends-and-coends]{профункторов},
где 0-ячейки — категории, 1-ячейки — профункторы, а 2-ячейки —
естественные преобразования. Композиция профункторов задавалась
коэндой.

\section{Монады}

К настоящему моменту вы должны быть хорошо знакомы с определением монады как
моноида в категории эндофункторов. Давайте пересмотрим это определение
с новым пониманием, что категория эндофункторов — это всего лишь одна
малая hom-категория эндо-$1$-ячеек в бикатегории $\Cat$. Мы
знаем, что это моноидальная категория: тензорное произведение происходит из
композиции эндофункторов. Моноид определяется как объект в
моноидальной категории --- здесь это будет эндофунктор $T$ ---
вместе с двумя морфизмами. Морфизмы между эндофункторами — естественные
преобразования. Один морфизм отображает моноидальную единицу --- тождественный
эндофунктор --- в $T$:
\[\eta \Colon I \to T\]
Второй морфизм отображает тензорное произведение $T \otimes T$ в
$T$. Тензорное произведение задаётся композицией эндофункторов, поэтому
получаем:
\[\mu \Colon T \circ T \to T\]

\begin{figure}[H]
  \centering
  \includegraphics[width=0.3\textwidth]{images/monad.png}
\end{figure}

\noindent
Мы узнаём их как две операции, определяющие монаду (они
называются \code{return} и \code{join} в Haskell), и мы знаем, что
законы моноида превращаются в законы монады.

Теперь давайте уберём все упоминания эндофункторов из этого определения. Начнём
с бикатегории $\cat{C}$ и выберем в ней $0$-ячейку $a$.
Как мы видели ранее, hom-категория $\cat{C}(a, a)$ — моноидальная
категория. Поэтому можем определить моноид в $\cat{C}(a, a)$,
выбрав $1$-ячейку $T$ и две $2$-ячейки:
\begin{align*}
  \eta & \Colon I \to T         \\
  \mu  & \Colon T \circ T \to T
\end{align*}
удовлетворяющие законам моноида. Мы называем \emph{это} монадой.

\begin{figure}[H]
  \centering
  \includegraphics[width=0.3\textwidth]{images/bimonad.png}
\end{figure}

\noindent
Это гораздо более общее определение монады, использующее только $0$-ячейки,
$1$-ячейки и $2$-ячейки. Оно сводится к обычной монаде при применении к
бикатегории $\Cat$. Но посмотрим, что происходит в других
бикатегориях.

Давайте построим монаду в $\cat{Span}$. Выберем $0$-ячейку, которая является
множеством, которое по причинам, которые скоро станут ясны, я назову
$\mathit{Ob}$. Далее выберем эндо-$1$-ячейку: спан из $\mathit{Ob}$ обратно
в $\mathit{Ob}$. Он имеет множество в вершине, которое я назову $\mathit{Ar}$,
оснащённое двумя функциями:
\begin{align*}
  \mathit{dom} & \Colon \mathit{Ar} \to \mathit{Ob} \\
  \mathit{cod} & \Colon \mathit{Ar} \to \mathit{Ob}
\end{align*}

\begin{figure}[H]
  \centering
  \includegraphics[width=0.3\textwidth]{images/spanmonad.png}
\end{figure}

\noindent
Назовём элементы множества $\mathit{Ar}$ ``стрелками''. Если я также
скажу вам называть элементы $\mathit{Ob}$ ``объектами'', вы можете получить
подсказку, к чему это ведёт. Две функции $\mathit{dom}$ и
$\mathit{cod}$ приписывают область определения и кообласть ``стрелке''.

Чтобы превратить наш спан в монаду, нам нужны две $2$-ячейки, $\eta$ и
$\mu$. Моноидальная единица в этом случае — тривиальный спан из
$\mathit{Ob}$ в $\mathit{Ob}$ с вершиной в $\mathit{Ob}$ и двумя тождественными
функциями. $2$-ячейка $\eta$ — это функция между вершинами
$\mathit{Ob}$ и $\mathit{Ar}$. Другими словами, $\eta$ приписывает
``стрелку'' каждому ``объекту''. $2$-ячейка в $\cat{Span}$ должна удовлетворять
условиям коммутации --- в данном случае:
\begin{align*}
  \mathit{dom} & \circ \eta = \id \\
  \mathit{cod} & \circ \eta = \id
\end{align*}

\begin{figure}[H]
  \centering
  \includegraphics[width=0.4\textwidth]{images/spanunit.png}
\end{figure}

\noindent
В компонентах это становится:
\[\mathit{dom}\ (\eta\ \mathit{ob}) = \mathit{ob} = \mathit{cod}\ (\eta\ \mathit{ob})\]
где $\mathit{ob}$ — ``объект'' в $\mathit{Ob}$. Другими словами,
$\eta$ приписывает каждому ``объекту'' ``стрелку'', чья область определения и
кообласть — этот ``объект''. Мы назовём эту специальную ``стрелку''
``тождественной стрелкой''.

Вторая $2$-ячейка $\mu$ действует на композицию спана
$\mathit{Ar}$ с самим собой. Композиция определяется как откат, поэтому
его элементы — пары элементов из $\mathit{Ar}$ --- пары
``стрелок'' $(a_1, a_2)$. Условие отката:
\[\mathit{cod}\ a_1 = \mathit{dom}\ a_2\]
Мы говорим, что $a_1$ и $a_2$ ``композируемы'', потому что
область определения одной — это кообласть другой.

\begin{figure}[H]
  \centering
  \includegraphics[width=0.5\textwidth]{images/spanmul.png}
\end{figure}

\noindent
$2$-ячейка $\mu$ — это функция, которая отображает пару композируемых
стрелок $(a_1, a_2)$ в одну стрелку $a_3$ из
$\mathit{Ar}$. Другими словами, $\mu$ определяет композицию стрелок.

Легко проверить, что законы монады соответствуют законам тождества и
ассоциативности для стрелок. Мы только что определили категорию (малую
категорию, заметьте, в которой объекты и стрелки образуют множества).

Итак, в общем, категория — это просто монада в бикатегории спанов.

Что удивительно в этом результате, так это то, что он ставит категории на одну
ногу с другими алгебраическими структурами, такими как монады и моноиды. Нет
ничего особенного в том, чтобы быть категорией. Это просто два множества и четыре
функции. На самом деле нам даже не нужно отдельное множество для объектов,
потому что объекты можно отождествить с тождественными стрелками (они находятся во
взаимно однозначном соответствии). Так что на самом деле это просто множество и несколько
функций. Учитывая ключевую роль, которую теория категорий играет во
всей математике, это очень смиряющее осознание.

\section{Задачи}

\begin{enumerate}
  \tightlist
  \item
        Выведите законы единицы и ассоциативности для тензорного произведения, определённого как
        композиция эндо-$1$-ячеек в бикатегории.
  \item
        Проверьте, что законы монады для монады в $\cat{Span}$ соответствуют
        законам тождества и ассоциативности в результирующей категории.
  \item
        Покажите, что монада в $\cat{Prof}$ — это функтор, тождественный на объектах.
  \item
        Что такое алгебра монады для монады в $\cat{Span}$?
\end{enumerate}

\section{Библиография}
\begin{enumerate}
  \tightlist
  \item
        \urlref{https://graphicallinearalgebra.net/2017/04/16/a-monoid-is-a-category-a-category-is-a-monad-a-monad-is-a-monoid/}{Блог Павела Собоциньского}.
\end{enumerate}
