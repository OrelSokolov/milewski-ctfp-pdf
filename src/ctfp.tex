% !TEX root = ctfp-print.tex
\input{half-title}
\frontmatter
\tableofcontents

\input{content/editor-note}
\chapter*{Предисловие}
\addcontentsline{toc}{chapter}{Предисловие}
\label{Preface}
\subfile{content/0.0/Preface}

\mainmatter

\part*{Часть Первая}
\addcontentsline{toc}{part}{Часть Первая}

\chapter{Категория: Суть Композиции}\label{category-the-essence-of-composition}
\subfile{content/1.1/category-the-essence-of-composition}

\chapter{Типы и Функции}\label{types-and-functions}
\subfile{content/1.2/types-and-functions}

\chapter{Категории Большие и Маленькие}\label{categories-great-and-small}
\subfile{content/1.3/categories-great-and-small}

\chapter{Категории Клейсли}\label{kleisli-categories}
\subfile{content/1.4/kleisli-categories}

\chapter{Произведения и Копроизведения}\label{products-and-coproducts}
\subfile{content/1.5/products-and-coproducts}

\chapter{Простые Алгебраические Типы Данных}\label{simple-algebraic-data-types}
\subfile{content/1.6/simple-algebraic-data-types}

\chapter{Функторы}\label{functors}
\subfile{content/1.7/functors}

\chapter{Функториальность}\label{functoriality}
\subfile{content/1.8/functoriality}

\chapter{Типы Функций}\label{function-types}
\subfile{content/1.9/function-types}

\chapter{Естественные Преобразования}\label{natural-transformations}
\subfile{content/1.10/natural-transformations}

\part*{Часть Вторая}
\addcontentsline{toc}{part}{Часть Вторая}

\chapter{Декларативное Программирование}\label{declarative-programming}
\subfile{content/2.1/declarative-programming}

\chapter{Пределы и Копределы}\label{limits-and-colimits}
\subfile{content/2.2/limits-and-colimits}

\chapter{Свободные Моноиды}\label{free-monoids}
\subfile{content/2.3/free-monoids}

\chapter{Представимые Функторы}\label{representable-functors}
\subfile{content/2.4/representable-functors}

\chapter{Лемма Йонеды}\label{the-yoneda-lemma}
\subfile{content/2.5/the-yoneda-lemma}

\chapter{Вложение Йонеды}\label{yoneda-embedding}
\subfile{content/2.6/yoneda-embedding}

\part*{Часть Третья}
\addcontentsline{toc}{part}{Часть Третья}

\chapter{Всё о Морфизмах}\label{all-about-morphisms}
\subfile{content/3.1/its-all-about-morphisms}

\chapter{Сопряжения}\label{adjunctions}
\subfile{content/3.2/adjunctions}

\chapter{Свободные/Забывающие Сопряжения}\label{free-forgetful-adjunctions}
\subfile{content/3.3/free-forgetful-adjunctions}

\chapter{Монады: Определение Программиста}\label{monads-programmers-definition}
\subfile{content/3.4/monads-programmers-definition}

\chapter{Монады и Эффекты}\label{monads-and-effects}
\subfile{content/3.5/monads-and-effects}

\chapter{Монады Категориально}\label{monads-categorically}
\subfile{content/3.6/monads-categorically}

\chapter{Комонады}\label{comonads}
\subfile{content/3.7/comonads}

\chapter{$F$-Алгебры}\label{f-algebras}
\subfile{content/3.8/f-algebras}

\chapter{Алгебры для Монад}\label{algebras-for-monads}
\subfile{content/3.9/algebras-for-monads}

\chapter{Концы и Коконцы}\label{ends-and-coends}
\subfile{content/3.10/ends-and-coends}

\chapter{Расширения Кана}\label{kan-extensions}
\subfile{content/3.11/kan-extensions}

\chapter{Обогащённые Категории}\label{enriched-categories}
\subfile{content/3.12/enriched-categories}

\chapter{Топосы}\label{topoi}
\subfile{content/3.13/topoi}

\chapter{Теории Ловера}\label{lawvere-theories}
\subfile{content/3.14/lawvere-theories}

\chapter{Монады, Моноиды и Категории}\label{monads-monoids-categories}
\subfile{content/3.15/monads-monoids-and-categories}

\backmatter

\appendix
\addcontentsline{toc}{part}{Приложения}
\input{index}

\makeatletter\@openrightfalse
\chapter*{Благодарности}\label{acknowledgments}
\addcontentsline{toc}{chapter}{Благодарности}
\noindent
Я хотел бы поблагодарить Эдварда Кметта и Гершома Базермана за проверку моей математики
и логики. Я благодарен многим добровольцам, которые исправили мои ошибки и улучшили книгу.

\vspace{1.0em}
\noindent
Я хотел бы поблагодарить Эндрю Саттона за переписывание моего кода концепта моноида на C++
согласно его и Бьярне Страуструпа последнему предложению.

\vspace{1.0em}
\noindent
Я благодарен Эрику Ниблеру за чтение черновика и предоставление
умной реализации \code{compose}, которая использует продвинутые возможности
C++14 для вывода типов. Я смог вырезать целый раздел
старомодной магии шаблонов, которая делала то же самое, используя type traits.
Хорошее избавление!

\chapter*{Колофон}\label{colophon}
\addcontentsline{toc}{chapter}{Колофон}
\lettrine[lraise=-0.03,loversize=0.08]{Э}{та книга} была скомпилирована \urlref{https://hmemcpy.com}{Игалем Табачником} путём преобразования оригинального текста Бартоша Милевски в формат \LaTeX{},
сначала извлекая оригинальные посты блога WordPress с помощью \urlref{https://mercury.postlight.com/web-parser/}{Mercury Web Parser}
для получения чистого HTML-содержимого, модифицируя и настраивая с помощью \urlref{https://www.crummy.com/software/BeautifulSoup/}{Beautiful Soup},
наконец, преобразуя в \LaTeX{} с помощью \urlref{https://pandoc.org/}{Pandoc}.

Шрифты: Linux Libertine для основного текста и Linux Biolinum для заголовков, оба от Филиппа Х. Полла. Моноширинный шрифт — Inconsolata,
созданный Рафом Левином и дополненный Димосфенисом Капонисом и Такаши Танигава в форме Inconsolata \acronym{LGC}. Шрифт обложки —
Alegreya, разработанный Хуаном Пабло дель Пералем.

Оригинальный дизайн макета книги и типографика выполнены Андресом Раба. Подсветка синтаксиса использует стиль ``GitHub'' для Pygments от
\urlref{https://github.com/hugomaiavieira/pygments-style-github}{Уго Майа Виеры}.
\ifdefined\OPTCustomLanguage{%
    \lettrine[lraise=-0.03,loversize=0.08]{Э}{та книга} была скомпилирована \urlref{https://hmemcpy.com}{Игалем Табачником} путём преобразования оригинального текста Бартоша Милевски в формат \LaTeX{},
сначала извлекая оригинальные посты блога WordPress с помощью \urlref{https://mercury.postlight.com/web-parser/}{Mercury Web Parser}
для получения чистого HTML-содержимого, модифицируя и настраивая с помощью \urlref{https://www.crummy.com/software/BeautifulSoup/}{Beautiful Soup},
наконец, преобразуя в \LaTeX{} с помощью \urlref{https://pandoc.org/}{Pandoc}.

Шрифты: Linux Libertine для основного текста и Linux Biolinum для заголовков, оба от Филиппа Х. Полла. Моноширинный шрифт — Inconsolata,
созданный Рафом Левином и дополненный Димосфенисом Капонисом и Такаши Танигава в форме Inconsolata \acronym{LGC}. Шрифт обложки —
Alegreya, разработанный Хуаном Пабло дель Пералем.

Оригинальный дизайн макета книги и типографика выполнены Андресом Раба. Подсветка синтаксиса использует стиль ``GitHub'' для Pygments от
\urlref{https://github.com/hugomaiavieira/pygments-style-github}{Уго Майа Виеры}.
\ifdefined\OPTCustomLanguage{%
    \lettrine[lraise=-0.03,loversize=0.08]{Э}{та книга} была скомпилирована \urlref{https://hmemcpy.com}{Игалем Табачником} путём преобразования оригинального текста Бартоша Милевски в формат \LaTeX{},
сначала извлекая оригинальные посты блога WordPress с помощью \urlref{https://mercury.postlight.com/web-parser/}{Mercury Web Parser}
для получения чистого HTML-содержимого, модифицируя и настраивая с помощью \urlref{https://www.crummy.com/software/BeautifulSoup/}{Beautiful Soup},
наконец, преобразуя в \LaTeX{} с помощью \urlref{https://pandoc.org/}{Pandoc}.

Шрифты: Linux Libertine для основного текста и Linux Biolinum для заголовков, оба от Филиппа Х. Полла. Моноширинный шрифт — Inconsolata,
созданный Рафом Левином и дополненный Димосфенисом Капонисом и Такаши Танигава в форме Inconsolata \acronym{LGC}. Шрифт обложки —
Alegreya, разработанный Хуаном Пабло дель Пералем.

Оригинальный дизайн макета книги и типографика выполнены Андресом Раба. Подсветка синтаксиса использует стиль ``GitHub'' для Pygments от
\urlref{https://github.com/hugomaiavieira/pygments-style-github}{Уго Майа Виеры}.
\ifdefined\OPTCustomLanguage{%
    \input{content/\OPTCustomLanguage/colophon}
  }
\fi
  }
\fi
  }
\fi

\chapter*{Уведомление о Copyleft}\label{copyleft}
\addcontentsline{toc}{chapter}{Уведомление о Copyleft}
\lettrine[lraise=-0.03,loversize=0.08]{Э}{та книга} является \textbf{свободной} и следует философии
\urlref{https://www.gnu.org/philosophy/free-sw.en.html}{свободного программного обеспечения}:
вы можете использовать эту книгу как хотите, исходный код доступен, вы можете распространять
эту книгу и можете распространять свою собственную версию. Это означает, что можете печатать её,
копировать, отправлять по электронной почте, загружать на веб-сайты, изменять, переводить, ремиксовать, удалять части
и рисовать по всей ней.

Эта книга распространяется по принципу Copyleft: если вы измените книгу и распространите свою собственную версию, вы также должны передать эти свободы её получателям.
Эта книга использует международную лицензию Creative Commons Attribution-ShareAlike 4.0
(\href{http://creativecommons.org/licenses/by-sa/4.0/}{\acronym{CC BY-SA 4.0}}).
\@openrighttrue\makeatother
\afterpage{\blankpage}
